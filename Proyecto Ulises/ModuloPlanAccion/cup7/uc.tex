%!TEX encoding = UTF-8 Unicode

\begin{UseCase}{CUP 7}{Administrar acciones}
	{
		
		Este caso de uso tiene como objetivo mostrar al actor todas las \cdtRef{gls:accion}{acciones} 
		registradas en el sistema asociadas a una meta, el actor podrá acceder a diversas operaciones como registrar, modificar y eliminar acciones.
	}
	%\UCitem{\DONEUC}{Edición}
	\UCitem{Versión}{1.0}
	\UCccsection{Administración de Requerimientos}	
	\UCitem{Autor}{Sergio Ramírez Camacho}	
	\UCccitem{Evaluador}{}
	\UCitem{Operación}{Administración}
	\UCccitem{Prioridad}{Alta}
	\UCccitem{Complejidad}{Media}
	\UCccitem{Volatilidad}{Alta}
	\UCccitem{Madurez}{Baja}
	\UCitem{Estatus}{Terminado}
	\UCitem{Fecha del último estatus}{2 de diciembre de 2014}
%% Copie y pegue este bloque tantas veces como revisiones tenga el caso de uso.
%% Esta sección la debe llenar solo el Revisor
% %--------------------------------------------------------
 	\UCccsection{Revisión Versión 1.0} % Anote la versión que se revisó.
 	\UCccitem{Fecha}{1-Dic} 
 	\UCccitem{Evaluador}{Nayeli Vega}
 	\UCccitem{Resultado}{Corregir}
 	\UCccitem{Observaciones}{
 		\begin{UClist}
% 			% PC: Petición de Cambio, describa el trabajo a realizar, si es posible indique la causa de la PC. Opcionalmente especifique la fecha en que considera razonable que se deba terminar la PC. No olvide que la numeración no se debe reiniciar en una segunda o tercera revisión.
 			\RCitem{PC1}{\DONE{Resumen: Liga rota de acciones}}{Fecha de entrega}
 			\RCitem{PC2}{\DONE{Postcondiciones: Liga rota en eliminar acción}}{Fecha de entrega}
 			\RCitem{PC3}{\DONE{Puntos de extensión: Liga rota de Eliminar acción}}{Fecha de entrega}
 			\RCitem{PC4}{\DONE{Interfaz: En comandos, liga rota a Eliminar acción}}{Fecha de entrega} 			
 		\end{UClist}		
 	}
% %--------------------------------------------------------


% %--------------------------------------------------------

	\UCsection{Atributos}
	\UCitem{Actor}{\cdtRef{actor:usuarioEscuela}{Coordinador del programa}}
	\UCitem{Propósito}{
		Administrar las acciones asociadas a una meta registradas en el sistema a través de una tabla de resultados.
	}
	\UCitem{Entradas}{
		Ninguna
	}
	\UCitem{Salidas}{
		\begin{UClist}
			\UCli \cdtRef{accion}{Acción}: \ioTabla{la \cdtRef{accion:descripcion}{Descripción}}{de acciones asociadas a la meta seleccionada}.
			\UCli \cdtIdRef{MSG2}{No existe información registrada por el momento}: Se muestra en la pantalla \cdtIdRef{IUP 7}{Administrar acciones} cuando no existen acciones registradas.
		\end{UClist}
	}

	\UCitem{Precondiciones}{
		\begin{UClist}
			\UCli {\bf Interna:} Que exista al menos una \cdtRef{gls:meta}{meta} registrada en el sistema.
		\end{UClist}
\UCli {\bf Interna:} Que la escuela se encuentre en estado \cdtRef{estado:planEdicion}{Plan de acción en edición}.
			\UCli {\bf Interna:} Que el periodo de registro de plan de acción se encuentre vigente.		
	}
	
	\UCitem{Postcondiciones}{
		\begin{UClist}
			\UCli {\bf Interna:} Se podrá registrar una acción por medio del caso de uso \cdtIdRef{CUP 8}{Registrar acción}.
			\UCli {\bf Interna:} Se podrá modificar una acción por medio del caso de uso \cdtIdRef{CUP 9}{Modificar acción}.
			\UCli {\bf Interna:} Se podrá eliminar una acción por medio del caso de uso \cdtIdRef{CUP 10}{Eliminar acción}.
		\end{UClist}
	}

	\UCitem{Reglas de \hspace{1 cm} negocio}{
		\begin{UClist}
			\UCli \cdtIdRef{RN-S6}{Títulos de las administraciones}: Indica cómo se debe mostrar el título del formulario.
		\end{UClist}
	}

	\UCitem{Errores}{
\UCli \cdtIdRef{MSG28}{Operación no permitida por estado de la entidad}: Se muestra en la pantalla en que se encuentre navegando el actor debido al estado en que se encuentra la escuela.
\UCli \cdtIdRef{MSG41}{Acción fuera del periodo}: Se muestra en la pantalla en que se encuentre navegando el actor indicando que la fecha no se encuentra dentro del periodo de registro de plan de acción.
	}

	\UCitem{Tipo}{
		Secundario, extiende del caso de uso \cdtIdRef{CUP 5}{Administrar metas}.}

% 	\UCitem{Fuente}{
% 		\begin{UClist}
% 			\UCli
% 		\end{UClist}
% 	}
	
\end{UseCase}
%-------------------------------------------------------%trayectoria Principal-----------------------------------------------
 \begin{UCtrayectoria}
    \UCpaso[\UCactor] Solicita administrar las acciones oprimiendo el botón \botAdm referente a la meta, en la pantalla \cdtIdRef{IUP 5}{Administrar metas}. %PENDIENTE
	\UCpaso[\UCsist] Verifica que la escuela se encuentre en  estado ``Plan de acción en edición''. \refTray{A}.
    \UCpaso[\UCsist] Verifica que la fecha actual se encuentre dentro del periodo definido por la SMAGEM para el registro del plan de acción. \refTray{B}.    
    \UCpaso[\UCsist] Busca la descripción de la meta seleccionada para mostrarla en la pantalla.
    \UCpaso[\UCsist] Busca la información de las acciones asociadas a la meta seleccionada. \refTray{C}
    \UCpaso[\UCsist] Verifica la línea de acción asociada a la acción para indicarla en el título como lo indica la regla de negocio \cdtIdRef{RN-S6}{Títulos de las administraciones}.
    \UCpaso[\UCsist] Muestra la información de las acciones en la pantalla \cdtIdRef{IUP 7}{Administrar acciones}. 
    \UCpaso[\UCactor] Administra las acciones a través de los botones: \cdtButton{Registrar}, \botAdm, \botEdit y \botKo. \label{cup7:Mostrar}
 \end{UCtrayectoria}

\begin{UCtrayectoriaA}[Fin del caso de uso]{A}{La escuela no se encuentra en el estado ``Plan de acción en edición''.}
    \UCpaso[\UCsist] Muestra el mensaje \cdtIdRef{MSG28}{Operación no permitida por estado de la entidad} en la pantalla en que se encuentre navegando el actor indicando que no puede administrar los objetivos del plan de acción debido a que la escuela no se encuentra en el estado ``Plan de acción en edición''. 
 \end{UCtrayectoriaA}
 
   \begin{UCtrayectoriaA}[Fin del caso de uso]{B}{La fecha actual se encuentra fuera del periodo definido por la SMAGEM para el registro del plan de acción}
    \UCpaso[\UCsist] Muestra el mensaje \cdtIdRef{MSG41}{Acción fuera del periodo} en la pantalla en que se encuentre navegando el actor indicando que no puede administrar los objetivos del plan de acción debido a que la fecha actual se encuentra fuera del periodo definido por la SMAGEM para el registro del plan de acción.
 \end{UCtrayectoriaA} 


\begin{UCtrayectoriaA}[Fin del caso de uso]{C}{No hay registros de acciones asociadas a la meta.}
    \UCpaso[\UCsist] Muestra el mensaje \cdtIdRef{MSG2}{No existe información registrada por el momento} en pantalla \cdtIdRef{IUP 7}{Administrar acciones} 
    indicando que aún no hay acciones registradas.
 \end{UCtrayectoriaA}
 

\subsection{Puntos de extensión}

  \UCExtensionPoint{El actor requiere registrar una acción}
	{Paso \ref{cup7:Mostrar}}
	{\cdtIdRef{CUP 8}{Registrar acción}}
	
  \UCExtensionPoint{El actor require modificar una acción}
	{Paso \ref{cup7:Mostrar}}
	{\cdtIdRef{CUP 9}{Modificar acción}}

  \UCExtensionPoint{El actor require eliminar una acción}
	{Paso \ref{cup7:Mostrar}}
	{\cdtIdRef{CUP 10}{Eliminar acción}}
	



 
