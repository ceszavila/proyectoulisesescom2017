\subsection{IUPB 1 Registrar meta de biodiversidad}
                     
\subsubsection{Objetivo}

   Esta pantalla permite al actor registrar una \cdtRef{gls:meta}{meta} de la línea de acción ``Biodiversidad''. Una vez finalizado el registro, el actor podrá registrar acciones asociadas a la meta.

	
\subsubsection{Diseño}

  En la figura ~\ref{IUPB 1} se muestra la pantalla ``Registrar meta de biodiversidad'',
  la cual permite al actor registrar una meta de la línea de acción ``Biodiversidad''. 
  El actor deberá ingresar los datos referentes a la meta, así como definir si la meta está enfocada al incremento de áreas verdes, 
  en este caso el sistema mostrará la información de la superficie del predio y el campo ``Metros cuadrados a incrementar'', como se muestra en la figura ~\ref{IUPB 1.1}.\\
  
  Si el actor selecciona que la meta no está enfocada al incremento de áreas verdes, el sistema mostrará la sección ``Información de la forestación y reforestación'',
  si el actor selecciona que la meta está enfocada a la plantación de árboles, el sistema mostrará el campo ``Número de árboles a plantar'', como se muestra en la pantalla ~\ref{IUPB 1.2}.\\
  
    
  Si el actor selecciona que la meta no está enfocada a la plantación de árboles, el sistema mostrará la sección ``Información de la capacitación y/o sensibilización'',
  si el actor selecciona que la meta está enfocada a dar capacitación y/o sensibilización, el sistema mostrará los campos relacionados y
  el actor deberá ingresar la información solicitada, así como seleccionar si la capacitación y/o sensibilización incluye dar a conocer
  el concepto de biodiversidad, como se muestra en la pantalla ~\ref{IUPB 1.3}.\\
    
  Si el actor selecciona que la meta no está enfocada a una capacitación y/o sensibilización, se mostrará la sección ``Cuantificar meta'' como se muestra en la pantalla ~\ref{IUPB 1.4}, el actor 
  deberá ingresar los datos referentes al valor a alcanzar.\\
 
  \IUfig[.9]{pantallas/planAccion/cupb1/iupb1}{IUPB 1}{Registrar meta de biodiversidad}
  \IUfig[.9]{pantallas/planAccion/cupb1/iupb1_1}{IUPB 1.1}{Registrar meta de biodiversidad: Incremento de áreas verdes}
  \IUfig[.9]{pantallas/planAccion/cupb1/iupb1_2}{IUPB 1.2}{Registrar meta de biodiversidad: Forestación y reforestación}
  \IUfig[.9]{pantallas/planAccion/cupb1/iupb1_3}{IUPB 1.3}{Registrar meta de biodiversidad: Capacitación y/o sensibilización}
  \IUfig[.9]{pantallas/planAccion/cupb1/iupb1_4}{IUPB 1.4}{Registrar meta de biodiversidad: Cuantificar meta}

  
  El sistema buscará las unidades disponibles para colocarlas como opciones en la lista desplegable ``Unidad'', si el actor selecciona la opción ``Otra'' de la lista desplegable, el sistema mostrará
  el campo ``Nueva unidad'' para que el actor defina la nueva unidad.\\
  
  Al oprimir el botón \cdtButton{Aceptar} el sistema validará la información ingresada y señalará los campos cuyos datos no cumplan con las reglas de negocio establecidas.\\
  
  Finalmente se mostrará el mensaje \cdtIdRef{MSG1}{Operación realizada exitosamente} en la pantalla \cdtIdRef{IUP 5}{Administrar metas}, para indicar que la información de la
  meta se ha registrado correctamente.
    
\subsubsection{Comandos}
\begin{itemize}
	\item \cdtButton{Aceptar}: Permite al actor confirmar el registro de la meta, dirige a la pantalla \cdtIdRef{IUP 5}{Administrar metas}.
	\item \cdtButton{Cancelar}: Permite al actor cancelar el registro de la meta, dirige a la pantalla \cdtIdRef{IUP 5}{Administrar metas}.
\end{itemize}


\subsubsection{Mensajes}

\begin{description}
	\item[\cdtIdRef{MSG1}{Operación realizada exitosamente}:] Se muestra en la pantalla \cdtIdRef{IUP 5}{Administrar metas} cuando la meta se ha registrado correctamente.
	\item[\cdtIdRef{MSG4}{No se encontró información sustantiva}:] Se muestra en la pantalla \cdtIdRef{IUP 5}{Administrar metas} cuando hace falta información referente a la unidad.
	\item[\cdtIdRef{MSG5}{Falta un dato requerido para efectuar la operación solicitada}:] Se muestra en la pantalla \cdtIdRef{IUPB 1}{Registrar meta de biodiversidad} cuando el actor no ingresó un dato requerido para realizar la operación.
	\item[\cdtIdRef{MSG6}{Formato incorrecto}:] Se muestra en la pantalla \cdtIdRef{IUPB 1}{Registrar meta de biodiversidad} cuando el formato de alguno de los datos ingresados es incorrecto.
	\item[\cdtIdRef{MSG7}{Se ha excedido la longitud máxima del campo}:] Se muestra en la pantalla \cdtIdRef{IUPB 1}{Registrar meta de biodiversidad} cuando el actor escribió un dato que excede el tamaño especificado por el sistema.
	\item[\cdtIdRef{MSG28}{Operación no permitida por estado de la entidad}:] Se muestra en la pantalla en que se encuentre navegando el actor debido al estado en que se encuentra la escuela.	
	\item[\cdtIdRef{MSG41}{Acción fuera del periodo}:] Se muestra en la pantalla en que se encuentre navegando el actor indicando que la fecha no se encuentra dentro del periodo de registro de plan de acción.
\end{description}
