
\label{ch:Antecedentes}
\section{Contexto}
El uso de las aplicaciones moviles cada día toma mas fuerza. Las ventajas que ofrecen los equipos inteligentes, como los smarthphones o las tabletas electrónicas, han resultado de suma importancia para diferentes ámbitos, siendo un hecho que el uso en la tecnología ha transformado de manera significativa el estilo de vida de las personas.
Así mismo resuelve las necesidades de los estudiantes en aspectos académicos, de tal modo que las aplicaciones móviles sirvan para dar un estilo de vida mas eficiente y practico en los alumnos

\section{Descomposición del Problema}
Nuestro equipo de trabajo identifico dos problemas a atacar
\begin{itemize}
	
	\item Confusión en la localización de las areas principales de la Escuela Superior de Computo: En la \textbf{Escuela Superior de Cómputo (ESCOM)} existen algunos problemas relacionados con las áreas existentes con las que cuenta, ademas de la manera en como están señalizadas.
	Este problema genera confusión, principalmente a los alumnos de primer semestre y personas visitantes a la ESCOM al no saber de manera correcta en donde se encuentra cada área perteneciente a la escuela, sin dejar a un lado que algunos de los alumnos de semestres mas arriba siguen sin saber de manera exacta donde se encuentran dichas áreas. 
	
	\item Los alumnos olvidan o extravían la identificación expirada por la ESCOM y no tienen acceso a la institución: En la \textbf{Escuela Superior de Cómputo (ESCOM)} existe vigilancia y un control de entrada controlada por seguridad PBI, los cuales tiene la instrucción de no permitir el acceso a la institución si los alumnos no presentan alguna identificación oficial.
	Esto hace que los alumnos pierdan tiempo al registrarse como personas visitantes y ovaciones genera una aglomeración de personas, importante en horas clave.
	
	
\end{itemize}

\section{Solución Propuesta}

Despues del análisis realizado trabajaremos en estos módulos:

\begin{itemize}
	\item \textbf{Localización de Áreas Principales} Mostrar las áreas principales de la ESCOM por medio de la utilización de mapas.
	
	\item \textbf{Identificación en Formato Digital} Mostrar la identificación brindada por la ESCOM de manera digital en un modulo, al girar de manera horizontal el dispositivo.	
\end{itemize}
