\subsection{IUS 25 Actualizar información del consumo de energía}

\subsubsection{Objetivo}

      En esta pantalla el \cdtRef{actor:usuarioEscuela}{Coordinador del programa} puede actualizar la información referente al consumo de energía en el caso de que esta haya cambiado.

\subsubsection{Diseño}

    En la figura~\ref{IUS 25} se muestra la pantalla ``Actualizar información del consumo de energía'', por medio de la cual se podrá actualizar la información referente al consumo de energía en el caso de que esta haya cambiado. El actor deberá ingresar la información solicitada en la pantalla, en la cual aparecerán los datos referentes al consumo e importe total.\\
    
    Si el actor selecciona la opción ``No'' en la pregunta ``¿La escuela cuenta con servicio de energía eléctrica?'' el sistema únicamente mostrará los botones \cdtButton{Aceptar} y \cdtButton{Cancelar} sin solicitar mas información. En caso contrario se mostrarán el resto de las preguntas como se muestra en la figura~\ref{IUS 25.1}\\
        
    Si se selecciona la opción ``Si'' en la pregunta ``¿Cuenta con recibos sobre el consumo de energía eléctrica en la escuela?'' se solicitará la frecuencia con la que recibe los recibos de cobro por concepto de energía como se muestra en la figura~\ref{IUS 25.2}. Si elige la opción ``Anual'' se mostrarán dos campos de texto para el ``Consumo anual'' en kilowatts-hora y el ``Importe anual'' en pesos.\\
    
    En el caso de que seleccione la opción ``Bimestral'', deberá seleccionar el bimestre al que corresponde cada uno de los recibos que registrará e ingresar el consumo de energía por bimestre en kilowatts-hora y el importe bimestral en pesos. Los registros por cada uno de los recibos bimestrales se mostrarán en una tabla, el consumo e importe por cada bimestre se sumará para mostrar los totales de ambos datos al final de la tabla como se muestra en la figura~\ref{IUS 25.3}. Lo mismo ocurrirá para las opciones ``Mensual'' y ``Semestral''. \\
    
    Si se selecciona la opción ``No'' en la pregunta ``¿Cuenta con recibos sobre el consumo de energía eléctrica en la escuela?'' se solicitará únicamente el consumo de energía e importe anual promedio, como se muestra en la figura~\ref{IUS 25.4}.\\
    
    Una vez que se haya ingresado toda la información solicitada para el registro de la información deberá oprimir el botón \cdtButton{Aceptar}, el sistema validará y registrará la información sólo si se han cumplido todas las reglas de negocio establecidas.\\
    
    Finalmente se mostrará el mensaje \cdtIdRef{MSG1}{Operación realizada exitosamente} en la pantalla \cdtIdRef{IUS 22}{Administrar avances de energía}, para indicar que la información referente al consumo de energía se ha actualizado exitosamente.
      
    \IUfig[.9]{pantallas/seguimiento/cus25/IUS25ActualizarInformacionEnergia.png}{IUS 25}{Actualizar información del consumo de energía}
    \IUfig[.9]{pantallas/seguimiento/cus25/IUS25ActualizarInformacionEnergia1.png}{IUS 25.1}{Actualizar información del consumo de energía: Cuenta con servicio de energía eléctrica}
    \IUfig[.9]{pantallas/seguimiento/cus25/IUS25ActualizarInformacionEnergia2.png}{IUS 25.2}{Actualizar información del consumo de energía: Cuenta con recibos del consumo de energía eléctrica}
    \IUfig[.9]{pantallas/seguimiento/cus25/IUS25ActualizarInformacionEnergia3.png}{IUS 25.3}{Actualizar información del consumo de energía: Recibos del consumo de energía eléctrica bimestrales}
    \IUfig[.9]{pantallas/seguimiento/cus25/IUS25ActualizarInformacionEnergia4.png}{IUS 25.4}{Actualizar información del consumo de energía: No cuenta con recibos del consumo de energía eléctrica}
    


\subsubsection{Comandos}
    \begin{itemize}
	\item \cdtButton{Aceptar}: Permite al actor actualizar la información referente al consumo de energía, dirige a la pantalla \cdtIdRef{IUS 22}{Administrar avances de energía}.
	\item \cdtButton{Cancelar}: Permite al actor cancelar la información referente al consumo de energía, dirige a la pantalla \cdtIdRef{IUS 22}{Administrar avances de energía}.
    \end{itemize}

\subsubsection{Mensajes}

    \begin{description}
	  \item [\cdtIdRef{MSG1}{Operación realizada exitosamente}:] Se muestra en la pantalla \cdtIdRef{IUS 22}{Administrar avances de energía} cuando la actualización de la información referente al consumo de energía se ha realizado correctamente.
	    
	  \item [\cdtIdRef{MSG5}{Falta un dato requerido para efectuar la operación solicitada}:] Se muestra en la pantalla \cdtIdRef{IUS 25}{ Actualizar información del consumo de energía} cuando no se ha ingresado un dato marcado como requerido.
	    
	  \item  [\cdtIdRef{MSG6}{Formato incorrecto}:] Se muestra en la pantalla \cdtIdRef{IUS 25}{Actualizar información del consumo de energía} cuando el tipo de dato ingresado no cumple con el tipo de dato solicitado en el campo.
	    
	  \item [\cdtIdRef{MSG7}{Se ha excedido la longitud máxima del campo}:] Se muestra en la pantalla \cdtIdRef{IUS 25}{ Actualizar información del consumo de energía} cuando se ha excedido la longitud de alguno de los campos.	    
	  
	  \item [\cdtIdRef{MSG28}{Operación no permitida por estado de la entidad}:] Se muestra en la pantalla \cdtIdRef{IUS 22}{Administrar avances de energía} indicando al actor que no se puede realizar la operación debido al estado en que se encuentra la escuela.
	    
	    \item [\cdtIdRef{MSG41}{Acción fuera del periodo}:] Se muestra en la pantalla \cdtIdRef{IUS 22}{Administrar avances de energía} para indicarle al actor que no puede realizar la operación debido a que la fecha actual se encuentra fuera del periodo definido por la SMAGEM para realizarla.
    \end{description}
