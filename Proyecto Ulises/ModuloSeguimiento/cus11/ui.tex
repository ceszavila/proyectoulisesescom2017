\subsection{IUS 11 Actualizar inventarios de flora y fauna}

\subsubsection{Objetivo}
    
    En esta pantalla el \cdtRef{actor:usuarioEscuela}{Coordinador del programa} puede acceder a la administración de los inventarios de flora y fauna.

\subsubsection{Diseño}

    En la figura~\ref{IUS 11} se muestra la pantalla ``Actualizar inventarios de flora y fauna'', por medio de la cual se podrá acceder a la administración de los inventarios de flora y fauna. El actor tendrá la facultad de acceder al registro y modificación de flora y fauna.
    
    \IUfig[.9]{pantallas/seguimiento/cus11/ius11}{IUS 11}{Actualizar inventarios de flora y fauna}

\subsubsection{Comandos}
    \begin{itemize}
    \item \botReg[Administrar fauna]: Permite al actor acceder a la administración del inventario de fauna, dirige a la pantalla \cdtIdRef{IUS 12}{Administrar inventario de fauna}.
    \item \botReg[Administrar flora]: Permite al actor acceder a la administración del inventario de flora, dirige a la pantalla \cdtIdRef{IUS 15}{Administrar inventario de flora}.
    \end{itemize}

\subsubsection{Mensajes}

    \begin{description}
    \item[\cdtIdRef{MSG28}{Operación no permitida por estado de la entidad}:] Se muestra en la pantalla \cdtIdRef{IUS 8}{Administrar avances de biodiversidad} indicando al actor que no se puede actualizar los inventarios debido al estado en que se encuentra la escuela.
    
    \item [\cdtIdRef{MSG41}{Acción fuera del periodo}:] Se muestra en la pantalla \cdtIdRef{IUS 8}{Administrar avances de biodiversidad} para indicarle al actor que no se puede actualizar los inventarios debido a que la fecha actual se encuentra fuera del periodo definido por la SMAGEM para realizar la acción.
\end{description}
