
\begin{UseCase}{CUS 1}{Administrar avances de objetivos}
    {
    Cuando se ha completado el registro de los objetivos para las líneas de acción, la escuela podrá acceder a este caso de uso, el cual permite visualizar los objetivos correspondientes a cada línea de acción y administrar los avances de sus metas y acciones.
    }
    % VERSION: Inicie con la versión 0.1 y contiúe 0.2, 0.3, hasta que sea aceptada por el evaluador será la versión 1.0. si surgen mas cambios continúe con 1.1, etc.
    \UCitem{Versión}{1.0}
    \UCccsection{Administración de Requerimientos}
    \UCitem{Autor}{Francisco Javier Ponce Cruz}
    \UCccitem{Evaluador}{}
    \UCitem{Operación}{Administrar}
    \UCccitem{Prioridad}{Alta}
    \UCccitem{Complejidad}{Baja}
    \UCccitem{Volatilidad}{Baja}
    \UCccitem{Madurez}{Alta}
    \UCitem{Estatus}{Terminado}
    \UCitem{Fecha del último estatus}{8 diciembre 2014}

    
%% Copie y pegue este bloque tantas veces como revisiones tenga el caso de uso.
%% Esta sección la debe llenar solo el Revisor
% %--------------------------------------------------------
 	\UCccsection{Revisión Versión 0.1} % Anote la versión que se revisó.
% 	% FECHA: Anote la fecha en que se terminó la revisión.
 	\UCccitem{Fecha}{9-Dic} 
% 	% EVALUADOR: Coloque el nombre completo de quien realizó la revisión.
 	\UCccitem{Evaluador}{Nayeli Vega}
% 	% RESULTADO: Coloque la palabra que mas se apegue al tipo de acción que el analista debe realizar.
 	\UCccitem{Resultado}{Corregir}
% 	% OBSERVACIONES: Liste los cambios que debe realizar el Analista.
 	\UCccitem{Observaciones}{
 		\begin{UClist}
% 			% PC: Petición de Cambio, describa el trabajo a realizar, si es posible indique la causa de la PC. Opcionalmente especifique la fecha en que considera razonable que se deba terminar la PC. No olvide que la numeración no se debe reiniciar en una segunda o tercera revisión.
 			\RCitem{PC1}{\TODO{Cambiar fecha}}{Fecha de entrega}
 			\RCitem{PC2}{\TODO{Cambiar precondición: Que la escuela se encuentre en estado Avance en edición}}{Fecha de entrega}
 			\RCitem{PC3}{\TODO{Agregar precondición de revisión de fecha actual}}{Fecha de entrega} 			
 			\RCitem{PC4}{\TODO{Agregar mensaje de error 41}}{Fecha de entrega} 			 			
 			\RCitem{PC5}{\TODO{Agregar a la trayectoria principal las verificaciones de estado y fecha}}{Fecha de entrega} 			 			 			
 			\RCitem{PC6}{\TODO{Agregar trayectorias alternativas}}{Fecha de entrega} 			 			 		
 			\RCitem{PC7}{\TODO{Interfaz: Verificar el título debe ser: IUS 1 }}{Fecha de entrega}  				 	
 			\RCitem{PC8}{\TODO{Interfaz: Redacción de diseño: Sugiero ... en la cual se muestran las líneas de acción y el objetivo asociado a cada una de ellas ...}}{Fecha de entrega}
 			\RCitem{PC9}{\TODO{Comandos: El objetivo no tiene avance, quienes tienen avance son las metas y las acciones}}{Fecha de entrega} 			 			 		 			
 			\RCitem{PC10}{\TODO{Interfaz: Agregar mensaje de error 41}}{Fecha de entrega} 			 			 			
 			
% 			\RCitem{PC2}{\TODO{Descripción del pendiente}}{Fecha de entrega}
% 			\RCitem{PC3}{\TODO{Descripción del pendiente}}{Fecha de entrega}
 		\end{UClist}		
 	}
% %--------------------------------------------------------

    \UCsection{Atributos}
    \UCitem{Actor(es)}{\cdtRef{actor:usuarioEscuela}{Coordinador del programa}}
    \UCitem{Propósito}{Visualizar y administrar los objetivos por línea de acción y acceder a la administración de las metas y acciones de cada uno}
    \UCitem{Entradas}{Ninguna.}
    \UCitem{Salidas}{ 
        \begin{UClist}
            \UCli \cdtRef{objetivo}{Objetivos}: \ioTabla{\cdtRef{objetivo:lineaAccion}{Línea de acción} y \cdtRef{objetivo:objetivoGeneral}{Objetivo general}}{de los objetivos registrados en el sistema}
        \end{UClist}
    }
    \UCitem{Precondiciones}{
	\begin{UClist}
	    \UCli {\bf Interna:} Que la escuela se encuentre en estado \cdtRef{estado:avanceEdicion}{Avance en edición}.
        \UCli {\bf Interna:} Que el periodo de registro de avances se encuentre vigente.
	\end{UClist}
    }
    \UCitem{Postcondiciones}{
	\begin{UClist}
	    \UCli {\bf Interna:} Se podrá acceder a la administración de los avances mediante los casos de uso: \cdtIdRef{CUS 2}{Administrar avance de metas de ambiente escolar}, \cdtIdRef{CUS 5}{Administrar avance de metas de consumo responsable}, \cdtIdRef{CUS 8}{Administrar avance de metas de biodiversidad}, \cdtIdRef{CUS 18}{Administrar avance de metas de agua}, \cdtIdRef{CUS 22}{Administrar avance de metas de energía}, \cdtIdRef{CUS 26}{Administrar avance de metas de residuos}.
	\end{UClist}
    }
    %Reglas de negocio: Especifique las reglas de negocio que utiliza este caso de uso
    \UCitem{Reglas de negocio}{Ninguna.}
    \UCitem{Errores}{
    \begin{UClist}
        \UCli \cdtIdRef{MSG28}{Operación no permitida por estado de la entidad}: Se muestra en la pantalla donde se encuentre navegando, indicando al actor que no se puede administrar los avances de objetivos ya que el Plan de acción no se encuentra aprobado.
    \end{UClist}
    }
    \UCitem{Tipo}{Primario}
\end{UseCase}


 \begin{UCtrayectoria}
    \UCpaso[\UCactor] Solicita administrar los avances de objetivos, seleccionando en el menú \cdtIdRef{MN2}{Menú del coordinador del programa} la opción ``Seguimiento y acreditación''.
    \UCpaso[\UCsist] Verifica que la escuela se encuentre en estado ``Avance en edición''. \refTray{A}.
    \UCpaso[\UCsist] Verifica que la fecha actual se encuentre dentro del periodo definido por la SMAGEM para administrar avances. \refTray{B}.
    \UCpaso[\UCsist] Verifica que existan registros de objetivos en el sistema. \refTray{C}.
    \UCpaso[\UCsist] Muestra la información de los objetivos en la pantalla \cdtIdRef{IUS 1}{Administrar avances de objetivos}.
    \UCpaso[\UCactor] Administra los avances de cada objetivo a través del botón \botAcciones. \label{cus1:Administrar}
 \end{UCtrayectoria}
 
 \begin{UCtrayectoriaA}[Fin del caso de uso]{A}{El Plan de acción no se encuentra en estado Aprobado}
    \UCpaso[\UCactor] Muestra el mensaje \cdtIdRef{MSG28}{Operación no permitida por estado de la entidad}  indicando al actor que no se puede administrar los avances de objetivos ya que el Plan de acción no se encuentra aprobado.
 \end{UCtrayectoriaA}

    \begin{UCtrayectoriaA}[Fin del caso de uso]{B}{La fecha actual se encuentra fuera del periodo definido por la SMAGEM para  administrar avances.}
    \UCpaso[\UCsist] Muestra el mensaje \cdtIdRef{MSG41}{Acción fuera del periodo} en la pantalla que se encuentre navegando, indicando al actor que no puede administrar los avances debido a que la fecha actual se encuentra fuera del periodo definido por la SMAGEM para realizar la acción. 
    \end{UCtrayectoriaA}

\begin{UCtrayectoriaA}[Fin del caso de uso]{C}{No hay registros de metas de objetivos para mostrar.}
    \UCpaso[\UCsist] Muestra el mensaje \cdtIdRef{MSG2}{No existe información registrada por el momento} en la pantalla \cdtIdRef{IUS 1}{Administrar avances de objetivos} indicando al actor que aún no hay objetivos registrados. 
 \end{UCtrayectoriaA}

\subsection{Puntos de extensión}

\UCExtensionPoint
{El Usuario desea administrar los avances de objetivo de Ambiente escolar}
{Paso \ref{cus1:Administrar} de la Trayectoria Principal}
{\cdtIdRef{CUS 2}{Administrar avance de metas de ambiente escolar}}

\UCExtensionPoint
{El Usuario desea administrar los avances de objetivo de Consumo responsable}
{Paso \ref{cus1:Administrar} de la Trayectoria Principal}
{\cdtIdRef{CUS 5}{Administrar avance de metas de consumo responsable}}

\UCExtensionPoint
{El Usuario desea administrar los avances de objetivo de Biodiversidad}
{Paso \ref{cus1:Administrar} de la Trayectoria Principal}
{\cdtIdRef{CUS 8}{Administrar avance de metas de biodiversidad}}

\UCExtensionPoint
{El Usuario desea administrar los avances de objetivo de Agua}
{Paso \ref{cus1:Administrar} de la Trayectoria Principal}
{\cdtIdRef{CUS 18}{Administrar avance de metas de agua}}

\UCExtensionPoint
{El Usuario desea administrar los avances de objetivo de Energía}
{Paso \ref{cus1:Administrar} de la Trayectoria Principal}
{\cdtIdRef{CUS 22}{Administrar avance de metas de energía}}

\UCExtensionPoint
{El Usuario desea administrar los avances de objetivo de Residuos}
{Paso \ref{cus1:Administrar} de la Trayectoria Principal}
{\cdtIdRef{CUS 26}{Administrar avance de metas de residuos}}