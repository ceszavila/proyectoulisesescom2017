% \begin{indicador}{ID}{Nombre del indicador}
% 	{Fórmula}
% 	{
% 		Descripción del indicador.
% 	} 	
% 	\INitem{Variable 1}{Unidad 1}{Descripción de variable 1}
% 	\INitem{Variable 2}{Unidad 2}{Descripción de variable 2}
% \end{indicador}

\subsection{Consumo responsable}

	Consumo responsable es un cambio de conducta que refleja una toma de conciencia sobre las condiciones actuales de las personas 
	y las comunidades en relación con su entorno y cómo ese cambio de conducta se refleja en acciones concretas relacionadas con diversos elementos socio-ambientales.\\

	En esta sección se muestran los indicadores para la línea de acción ``Consumo responsable'' propuestos para el programa.

%------------------------------------------------------------------------------------------------
\begin{indicador}{IC 1}{Porcentaje de personas que consumen alimentos frescos cuando están en la escuela}
	{$\%AF = \frac{PAF_1 + PAF_2 + ... + PAF_n}{PE_1 + PE_2 + ... + PE_n}$} 
	{
		La fuente de datos de este indicador son las respuestas a los cuestionarios aplicados, de esta línea de acción. Se calcula dividiendo el número de respuestas afirmativas entre el número total de cuestionarios contestados. \\
		
		El aumento del consumo de alimentos frescos está directamente relacionado con la salud y la calidad de vida,
		por ello la medida podría considerarse un  indicador de sustentabilidad. \\

		Cabe mencionar que un alimento fresco se define como aquel sin procesar o conservar, también se conoce como alimento natural.
	} 	
	\INitem{$\cdtRef{gls:porcentajeAF}{\%AF}$}{$\%$}{\cdtRef{gls:porcentajeAF}{Porcentaje de personas que declararon consumir alimentos frescos}.}
	\INitem{$PAF_1, PAF_2, ..., PAF_n$}{$personas$}{Número de \cdtRef{consumo:consumoAlimentosFrescos}{personas que consumen alimentos frescos}.}
	\INitem{$PE_1, PE_2, ..., PE_n$}{$personas$}{Número de \cdtRef{consumo:personasEncuestadas}{personas encuestadas}.}
\end{indicador}
%------------------------------------------------------------------------------------------------
\begin{indicador}{IC 2}{Porcentaje de compras de bienes de bajo impacto ambiental y/o reciclados para la escuela (compras verdes) por año}
	{$\%CV = \frac{CVP + CVL + CVC}{P + L + E} * 100$}
	{
		Se define como compras verdes, a aquellas adquisiciones de bienes o servicios que generan menor impacto ambiental que otros con la misma función. 
		Por ejemplo, equipos eléctricos con bajo consumo de energía, muebles de baño de bajo consumo de agua, materiales de limpieza no tóxicos, papel reciclado, plumones base agua, etc. 
		Estos productos tienen etiquetas que los identifican como de bajo consumo, ecológicos, amigables con el ambiente o reciclados.\\

		Para calcular el indicador, se deberán contabilizar las compras de tres tipos de insumos:
		\begin{itemize}
		 \item De materiales de papelería
		 \item De materiales de limpieza
		 \item De equipos electrónico
		\end{itemize}
	} 	
	\INitem{$\cdtRef{gls:comprasVerdes}{\%CV}$}{$ $}{Total de \cdtRef{gls:comprasVerdes}{compras verdes} que realiza la escuela.}
	\INitem{$CVP$}{$ $}{Número de \cdtRef{consumo:comprasReciclados}{compras de papelería reciclados}.}
	\INitem{$CVL$}{$ $}{Número de \cdtRef{consumo:comprasNoToxicos}{compras de productos de limpieza no tóxicos}.}
	\INitem{$CVC$}{$ $}{Número de \cdtRef{consumo:comprasBajoConsumo}{compras de bajo consumo de energía}.}
	\INitem{$P$}{$ $}{Total de \cdtRef{consumo:comprasAnioPapeleria}{compras al año de papelería}.}
	\INitem{$L$}{$ $}{Total de \cdtRef{consumo:comprasAnioLimpieza}{compras al año de productos de limpieza}.}
	\INitem{$E$}{$ $}{Total de \cdtRef{consumo:comprasAnioElectronicos}{compras al año de equipo electrónico}.}
\end{indicador}
