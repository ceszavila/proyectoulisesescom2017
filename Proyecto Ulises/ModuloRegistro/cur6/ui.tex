\subsection{IUR 6 Completar información escolar}

\subsubsection{Objetivo}

     Esta pantalla permite al actor  \cdtRef{actor:usuarioEscuela}{Coordinador del programa} completar el registro de la información escolar en el sistema.

\subsubsection{Diseño}

   En la figura~\ref{IUR 6} se muestra la pantalla ``Completar información escolar'', en la cual el actor deberá ingresar los datos al sistema para completar el registro de la escuela.\\
    
    Una vez ingresada la información solicitada en la pantalla, deberá oprimir el botón \cdtButton{Aceptar}, posteriormente  el sistema validará %se mostrará el mensaje \cdtIdRef{MSG3}{Confirmación de envío de información} \TODO{poner el mensaje en la sección de mensajes si no está y colocar la liga adecuadamente} como pantalla emergente \TODO{liga a la pantalla emergente}, en la cual deberá confirmar el envío de la información sólo 
    si se han cumplido todas las reglas de negocio establecidas.\\
    
    Finalmente se mostrará el mensaje \cdtIdRef{MSG1}{Operación realizada exitosamente} en la pantalla \cdtIdRef{IUR 5}{Adminstrar información escolar} para indicar que la información se ha registrado exitosamente.
    
	En caso de haber seleccionado una región que no pertenece al municipio de la escuela, se mostrará como pantalla emergente el mensaje \cdtIdRef{MSG18}{Error en la región}.
    
    \IUfig[.75]{pantallas/registro/IUR6}{IUR 6}{Completar información escolar}

%    \IUfig[.6]{pantallas/registro/IUR6_1}{IUR 6.1}{Notificación de la región}

\subsubsection{Comandos} 
    \begin{itemize}
	\item \cdtButton{Aceptar}: Permite al actor completar la información de la escuela en el sistema, dirige a la pantalla \cdtIdRef{IUR 5}{Administrar información escolar}.
    \item \cdtButton{Cancelar}: Permite al actor cancelar el completado de la información de la escuela, dirige a la pantalla \cdtIdRef{IUR 5}{Administrar información escolar}.
    \end{itemize}


\subsubsection{Mensajes} 

    \begin{description}
    \item[\cdtIdRef{MSG1}{Operación realizada exitosamente}:] Se muestra en la pantalla \cdtIdRef{IUR 5}{Administrar información escolar} cuando los datos de la escuela han sido registrados exitosamente.
	\item [\cdtIdRef{MSG3}{Superficies del predio}:] Se muestra en la pantalla \cdtIdRef{IUR 6}{Completar información escolar} para notificar al actor que la superficie construida que ha ingresado supera la superficie total del predio.      
	\item [\cdtIdRef{MSG4}{No se encontró información sustantiva}:] Se muestra en la pantalla \cdtIdRef{IUR 6}{Completar información escolar} cuando no hay información referente a la escuela.
    \item[\cdtIdRef{MSG5}{Falta un dato requerido para efectuar la operación solicitada}:] Se muestra en la pantalla \cdtIdRef{IUR 6}{Completar información escolar} cuando al menos un dato requerido no ha sido ingresado.

    \item[\cdtIdRef{MSG6}{Formato incorrecto}:] Se muestra en la pantalla \cdtIdRef{IUR 6}{Completar información escolar} especificando el dato cuyo valor no cumple con el tipo de dato definido en el diccionario de datos.

    \item[\cdtIdRef{MSG7}{Se ha excedido la longitud máxima del campo}:] Se muestra en la pantalla \cdtIdRef{IUR 6}{Completar información escolar} especificando el dato cuya longitud excede el tamaño máximo permitido.

    \item[\cdtIdRef{MSG18}{Error en la región}:] Se muestra en la pantalla \cdtIdRef{IUR 6}{Completar información escolar} para notificar al actor que la escuela no se encuentra en un municipio asociado a la región seleccionada.

    \end{description}
