\subsection{IUR 9 Administrar responsable del programa}

\subsubsection{Objetivo}

% Explicar el objetivo para el que se construyo la interfaz, generalmente es la descripción de la actividad a desarrollar, como Seleccionar grupos para inscribir materias de un alumno, controlar el acceso al sistema mediante la solicitud de un login y password de los usuarios, etc.

	En esta pantalla el actor \cdtRef{actor:usuarioEscuela}{Coordinador del programa} puede administrar al responsable del programa, sirve como punto de acceso para registrarlo en caso de no existir, así como modificar o visualizar su información para el caso contrario.

\subsubsection{Diseño}

% Presente la figura de la interfaz y explique paso a paso ``a manera de manual de usuario'' como se debe utilizar la interfaz. No olvide detallar en la redacción los datos de entradas y salidas. Explique como utilizar cada botón y control de la pantalla, para que sirven y lo que hacen. Si el Botón lleva a otra pantalla, solo indique la pantalla y explique lo que pasará cuando se cierre dicha pantalla (la explicación sobre el funcionamiento de la otra pantalla estará en su archivo correspondiente).

%     Para el diseño de la interfaz de usuario se tomaron en cuenta dos escenarios	 de interacción, el primero donde el sistema ya cuenta con un responsable del programa previamente registrado y el segundo cuando no existe un responsable del programa en el sistema.

    En la figura~\ref{IUR 9} se muestra la pantalla ``Administrar responsable del programa'', en este escenario, el responsable del programa no ha sido registrado, al no existir registro se muestra la opción de registro por medio del botón \cdtButton{Registrar responsable} para realizar el registro pertinente.
    %toma en cuenta el segundo escenario de interacción y permite registrar al responsable del programa teniendo habilitado el botón \cdtButton{Registrar responsable}.

    \IUfig[.9]{pantallas/registro/IUR9-1}{IUR 9}{Administrar responsable del programa}

    En la figura~\ref{IUR 9.1} se muestra la pantalla ``Administrar responsable del programa'', en la cual se muestra el escenario en que ya existe un responsable del programa registrado en el sistema al cual se le puede administrar %toma en cuenta el primer escenario de interacción   y permite administrar al responsable del programa 
    a través de una tabla de resultados donde se muestra el nombre o nombres del responsable, el primer apellido, el segundo apellido y el puesto que desempeña. El botón \cdtButton{Registrar responsable} no aparece en este escenario debido a que ya se encuentra registrado el responsable del programa, a su vez se muestra la frase ``Ya se ha registrado al responsable del programa''.%de esta pantalla pues sólo puede existir un responsable del programa registrado en el sistema.

    \IUfig[.9]{pantallas/registro/IUR9}{IUR 9.1}{Administrar responsable del programa (Con registro)}

\subsubsection{Comandos}
\begin{itemize}
	\item \cdtButton{Registrar}: Permite al actor registrar un responsable del programa en caso de no existir, dirige a la pantalla  \cdtIdRef{IUR 10}{Registrar responsable del programa.}
	\item \botEdit[Modificar responsable del programa]: Permite al actor modificar la información correspondiente al responsable del programa, dirige a la pantalla \cdtIdRef{IUR 11}{Modificar responsable del programa.} 
	\item \botV[Visualizar responsable del programa]: Permite al actor visualizar la información completa del responsable del programa, dirige a la pantalla \cdtIdRef{IUR 12}{Visualizar responsable del programa}
\end{itemize}

\subsubsection{Mensajes}

	
\begin{description}
	\item[\cdtIdRef{MSG2}{No existe información registrada por el momento}:] Se muestra este mensaje de error en la pantalla \cdtIdRef{IUR 9}{Administrar responsable del programa} cuando aún no se ha registrado un responsable del programa.
\end{description}
