%!TEX encoding = UTF-8 Unicode

% ESTA SECCION LA DEBE LLENAR SOLO EL ANALISTA.
% ID: Asegurese de que el ID del Caso de uso sea único.
% Nombre: Aseurese de que esté escrito de la forma: VERBO + SUSTANTIVO + ALGO
\begin{UseCase}{CUR 11}{Modificar responsable del programa}
	{
		Este caso de uso permite modificar la información del responsable del programa si se ha detectado algún error en ella para así actualizar la información del mismo. Una vez realizados los cambios en la información, el sistema valida y actualiza los datos. 
	}
	% VERSION: Inicie con la versión 0.1 y contiúe 0.2, 0.3, hasta que sea aceptada por el evaluador será la versión 1.0. si surgen mas cambios continúe con 1.1, etc.
	\UCitem{Versión}{1.0}
	\UCccsection{Administración de Requerimientos}
	\UCitem{Autor}{Angélica Madrid Jiménez}
	\UCccitem{Evaluador}{José David Ortega Pacheco}
	\UCitem{Operación}{Modificación}
	\UCccitem{Prioridad}{Alta}
	\UCccitem{Complejidad}{Media}
	\UCccitem{Volatilidad}{Baja}
	\UCccitem{Madurez}{Alta}
    \UCitem{Estatus}{Terminado}
    \UCitem{Fecha del último estatus}{5 de Noviembre del 2014}



%% Copie y pegue este bloque tantas veces como revisiones tenga el caso de uso.
%% Esta sección la debe llenar solo el Revisor
% %--------------------------------------------------------
% 	\UCccsection{Revisión Versión XX} % Anote la versión que se revisó.
% 	% FECHA: Anote la fecha en que se terminó la revisión.
% 	\UCccitem{Fecha}{Fecha en que se termino la revisión} 
% 	% EVALUADOR: Coloque el nombre completo de quien realizó la revisión.
% 	\UCccitem{Evaluador}{Nombre de quien revisó}
% 	% RESULTADO: Coloque la palabra que mas se apegue al tipo de acción que el analista debe realizar.
% 	\UCccitem{Resultado}{Corregir, Desechar, Rehacer todo, terminar.}
% 	% OBSERVACIONES: Liste los cambios que debe realizar el Analista.
% 	\UCccitem{Observaciones}{
% 		\begin{UClist}
% 			% PC: Petición de Cambio, describa el trabajo a realizar, si es posible indique la causa de la PC. Opcionalmente especifique la fecha en que considera razonable que se deba terminar la PC. No olvide que la numeración no se debe reiniciar en una segunda o tercera revisión.
% 			\RCitem{PC1}{\TODO{Prescondiciones que exista un responsable del programa ya registrado}}{Fecha de entrega}
% 			\RCitem{PC1}{\TODO{Predncición: que se haya iniciado sesión}}{Fecha de entrega}
% 			\RCitem{PC2}{\TODO{Errores. Comenzar por el mensaje y seguí con la descripción del porque del mensaje}}{Fecha de entrega}
% 			\RCitem{PC3}{\TODO{Tipo: Secundario}}{Fecha de entrega}
% 			\RCitem{PC3}{\TODO{En la trayectoria principal, número 6. Separar un paso por cada uno de las reglas de negocio}}{Fecha de entrega} 			
%			\RCitem{PC3}{\TODO{Trayectoria Alterna A, cambiar bien la referencia a al pantalla de retorno}}{Fecha de entrega} 			 			
%			\RCitem{PC3}{\TODO{Trayectoria Alterna B, Fatla punto de retorno}}{Fecha de entrega} 		
%			\RCitem{PC3}{\TODO{PANTALLA: Sólamente hay un actor que es el Coordinador del programa}}{Fecha de entrega} 					
%			\RCitem{PC3}{\TODO{PANTALLA: Diseño: Datos previamente registrados}}{Fecha de entrega} 					
%			\RCitem{PC3}{\TODO{PANTALLA: Comandos: Aceptar: permite al usuario modificar en el sistema la información del Reponsable del programa}}{Fecha de entrega} 											
%			\RCitem{PC3}{\TOCHK{Precondición: Que el usuario haya iniciado sesión}}{Fecha de entrega} 				
%			\RCitem{PC3}{\TOCHK{Errores: Falta el error de formato de correo electróncio }}{Fecha de entrega} 		
%			\RCitem{PC3}{\TOCHK{Tipo: De donde extiende}}{Fecha de entrega} 														
% 		\end{UClist}		
%	}
% %--------------------------------------------------------

	\UCsection{Atributos}
	\UCitem{Actor(es)}{\cdtRef{actor:usuarioEscuela}{Coordinador del programa}}
	\UCitem{Propósito}{Modificar la información específica del responsable del programa si se ha identificado algún error o se requiere actualizar la información del mismo.}
	\UCitem{Entradas}{
		\begin{UClist}
			\UCli \cdtRef{persona:nombre}{Nombre(s)}: \ioEscribir.
			\UCli \cdtRef{persona:primerApellido}{Primer apellido} \ioEscribir.
			\UCli \cdtRef{persona:segundoApellido}{Segundo apellido} \ioEscribir.
			\UCli \cdtRef{responsable:cveEmpleado}{Clave de empleado} \ioEscribir.
			\UCli \cdtRef{responsable:puesto}{Puesto}: \ioEscribir.
			\UCli \cdtRef{persona:correo}{Correo electrónico} \ioEscribir.
			\UCli \cdtRef{empleado:telefono}{Teléfono}: \ioEscribir.
			\UCli \cdtRef{empleado:extension}{Extensión}: \ioEscribir.
		\end{UClist}
	}
	% SALIDAS: Liste y referencíe los datos de salida o resultados del sistema, Especifíque el dispositivo en donde se presentarán las salidas: pantalla, impresora, otro sistema, brazo mecánico, etc. 
	\UCitem{Salidas}{
		\begin{UClist} 
		\UCli \cdtRef{persona:nombre}{Nombre(s)}: \ioObtener.
			\UCli \cdtRef{persona:primerApellido}{Primer apellido} \ioObtener.
			\UCli \cdtRef{persona:segundoApellido}{Segundo apellido} \ioObtener.
			\UCli \cdtRef{responsable:cveEmpleado}{Clave de empleado} \ioObtener.
			\UCli \cdtRef{responsable:puesto}{Puesto}: \ioObtener.
			\UCli \cdtRef{persona:correo}{Correo electrónico} \ioObtener.
			\UCli \cdtRef{empleado:telefono}{Teléfono}: \ioObtener.
			\UCli \cdtRef{empleado:extension}{Extensión}: \ioObtener.
		       \UCli \cdtIdRef{MSG1}{Operación realizada exitosamente}: Se muestra en la pantalla \cdtIdRef{IUR 9}{Administrar responsable del programa} cuando la modificación se realizó correctamente.
		\end{UClist}
	}
	\UCitem{Precondiciones}{
		\begin{UClist}
% 			\UCli {\bf Interna:} Que el coordinador del programa haya iniciado sesión.
			\UCli {\bf Interna:} Que exista responsable del programa registrado.
		\UCli {\bf Interna:} Que la escuela se encuentre en estado \cdtRef{estado:inscrita}{inscrita}.	    
			\UCli {\bf Interna:} Que el periodo de registro de escuelas se encuentre vigente.			
		\end{UClist}
	}
	
	\UCitem{Postcondiciones}{
		\begin{UClist}
		    \UCli {\bf Interna:} Los cambios registrados para el responsable del programa se mostrarán en la pantalla \cdtIdRef{CUR 9}{Administrar responsable del programa}
		 \end{UClist}
	}
	
	\UCitem{Reglas de negocio}{
		\begin{UClist}
			\UCli \cdtIdRef{RN-S1}{Información correcta}. Verifica que la información introducida sea correcta.
			\UCli \cdtIdRef{RN-S2}{Formato de correo electrónico}. Verifica que el correo electrónico proporcionado tenga un formato correcto de escritura.
			%%%%%%%%%%%%%%%REGLA DEL NEGOCIO DE LONGITUD MÁXIMA%%%%%%%%%%%%%%
		\end{UClist}
			
	}


	\UCitem{Errores}{
	
		\begin{UClist}
			\UCli \cdtIdRef{MSG5}{Falta un dato requerido para efectuar la operación solicitada}: Se muestra en la pantalla \cdtIdRef{IUR 11}{Modificar responsable del programa} cuando no se ha ingresado un dato requerido.
			\UCli \cdtIdRef{MSG6}{Formato incorrecto}: Se muestra en la pantalla \cdtIdRef{IUR 11}{Modificar responsable del programa} indicando que el valor ingresado es incorrecto ya que no cumple con el formato especificado.
			\UCli \cdtIdRef{MSG7}{Se ha excedido la longitud máxima del campo}: Se muestra en la pantalla \cdtIdRef{IUR 11}{Modificar responsable del programa}  indicando que la operación de registro no puede realizarse debido a que la información no es correcta y señala el campo que presenta el tipo de dato erróneo. 
			\UCli \cdtIdRef{MSG16}{Error en formato de correo electrónico}: Se muestra en la pantalla \cdtIdRef{IUR 11}{Modificar responsable del programa} indicando que el correo electrónico proporcionado no cumple con un formato válido.
		\end{UClist}
     \UCli \cdtIdRef{MSG28}{Operación no permitida por estado de la entidad}: Se muestra sobre la pantalla \cdtIdRef{IUR 9}{Administrar responsable del programa} indicando al actor que no puede modificar al responsable del programa debido al estado en que se encuentra la escuela.
		\UCli	\cdtIdRef{MSG41}{Acción fuera del periodo}: Se muestra sobre la pantalla \cdtIdRef{IUR 9}{Administrar responsable del programa} para indicarle al actor que no puede modificar al responsable del programa debido a que la fecha actual se encuentra fuera del periodo definido por la SMAGEM para realizar el registro de escuelas.
	}

	\UCitem{Tipo}{Secundario, extiende del caso de uso \cdtIdRef{CUR 9}{Administrar responsable del programa}.}

	\UCitem{Fuente}{
    \begin{UClist}
      \UCli Minuta de la reunión \cdtIdRef{M-3TR}{Toma de requerimientos}.
    \end{UClist}
  }

 \end{UseCase}

 % La trayectoria principal debe describir los pasos del caso correcto simple completo. Esto es:
 % Correcto: Describe el caso mas común de la ejecución correcta del Caso de uso.
 % Simple: No repite pasos ni especifica iteraciones, no abunda en errores.
 % Completo: Anque no abunda en errores especifica todas las validaciones, verificaciones, comparaciones y cálculos que debe realizar el sistema durante la trayectoria.
 % FACTORES CRITICOS:
 %-------------------
 % 1.- Cada paso del actor debe estar redactado ``en lo posible'' con dos partes:
 %   - La primera parte especifica ``lo que el Actor cree que está haciendo'' por ejemplo: ``Solicita su registro''.
 %   - La segunda especifica ``la acción exacta que realiza en el sistema''. por ejemplo ``oprime el botón \IUButton{Registrar} de la \cdtIdRef{IU4}{Datos personales}''.
 % En conjunto la redacción de los pasos se debe ver de la siguiente forma: 
 %
 %``El actor solicita su registro oprimiendo el botón \cdtButton{Registrar} de la \cdtIdRef{IU4}{Datos personales}.''
 %
 % Para los pasos del sistema aplica lo mismo solo para los pasos asociados con una Regla de Negocio o un cambio en la Interfaz de Usuario. Por ejemplo:
 % ``El sistema busca los vehículos disponibles.''
 % ``El sistema verifica que las tareas correspondan, de acuerdo a lo especificado en la \cdtIdRef{RN34}{Compatibilidad entre tareas de un mismo proyecto}''
 % ``El sistema muestra los datos personales del Alumno mediante la \cdtIdRef{IU23}{Datos del Alumno}''
 %
 % 2.- Cada paso debe comenzar con un verbo, jamás con un si, mientras, cuando, entonces, etc. utilice: Escribe, registra, calcula, verifica, muestra, lista, selecciona, asocia, filtra, busca, etc.
 %
 % 3.- Ponga especial atención en las verificaciones y detección de errores
 %
 % 4.- Describa con un paso que diga ``el sistema busca ...'' antes de mostrar una pantalla que tiene listas que se llenan desde un catálogo.
 %
 % 5.- Referencie las reglas de negocio en las validaciones que correspondan.
 %
 % 6.- Referencie la Interfaz de Usuario correspondiente cuando el sistema muestre un dato o mensaje.
 % 7.- El Caso de Uso debe iniciar con el actor justo después del paso en el que se marque el Punto de Extensión o Inclusión del caso de uso proveniente. Considere que al inicio del Caso de Uso cuando es secundario se debe especificar la(s) interfaz(es) con la que se está trabajando y debe coincidir con la Interfaz activa al momento de la Extensión o Inclusión.
 %%%%TRAYECTORIA PRINCIPAL%%%
 
 \begin{UCtrayectoria}
    \UCpaso[\UCactor] Solicita modificar la información del responsable del programa oprimiendo el botón \botEdit de la pantalla \cdtIdRef{IUR 9}{Administrar responsable del programa}

    \UCpaso[\UCsist] Verifica que la escuela se encuentre es estado ``Inscrita''. \refTray{A}
    \UCpaso[\UCsist] Verifica que la fecha actual se encuentre dentro del periodo definido por parte de la SMAGEM para el registro de escuelas. \refTray{B}
    
    \UCpaso[\UCsist] Busca la información del responsable del programa asociado a la escuela en el sistema. 
    \UCpaso[\UCsist] Muestra la pantalla \cdtIdRef{IUR 11}{Modificar responsable del programa} con los datos del responsable del programa.
    \UCpaso[\UCactor] Actualiza los datos del responsable del programa en la pantalla \cdtIdRef{IUR 11}{Modificar responsable del programa}. \label{cur11:Acciones} 
    \UCpaso[\UCactor] Solicita guardar la información modificada del responsable del programa oprimiendo el botón \cdtButton{Aceptar} de la pantalla \cdtIdRef{IUR 11}{Modificar responsable del programa} . \refTray{C}

    \UCpaso[\UCsist] Verifica que la escuela se encuentre es estado ``Inscrita''. \refTray{A}
    \UCpaso[\UCsist] Verifica que la fecha actual se encuentre dentro del periodo definido por parte de la SMAGEM para el registro de escuelas. \refTray{B}    
    
    \UCpaso[\UCsist] Verifica que los datos ingresados proporcionen información correcta de acuerdo a la regla de negocio \cdtRef{RN-S1}{RN-S1 Información correcta}. \refTray{D} , \refTray{E}, \refTray{F}
   \UCpaso[\UCsist]Verifica el formato del correo electrónico como lo indica la regla de negocio \cdtRef{RN-S2}{RN-S2 Formato de correo electrónico} \refTray{G} %%El usuario no ingresó todos los datos solicitados/no ingresó datos correctos/Duplicidad de datos
    \UCpaso[\UCsist] Actualiza los datos del responsable del programa en el sistema.
    \UCpaso[\UCsist] Muestra el mensaje \cdtIdRef{MSG1}{Operación realizada exitosamente} en la pantalla \cdtIdRef{IUR 9}{Administrar responsable del programa} especificando que los datos del responsable se han modificados de manera exitosa.

 \end{UCtrayectoria}
 %%%TRAYECTORIAS ALTERNATIVAS%%%

 \begin{UCtrayectoriaA}[Fin del caso de uso]{A}{La escuela no se encuentra en el estado ``Inscrita''}
    \UCpaso[\UCsist] Muestra el mensaje \cdtIdRef{MSG28}{Operación no permitida por estado de la entidad} en la pantalla \cdtIdRef{IUR 9}{Administrar responsable del programa} indicando al actor que no puede modificar al responsable del programa debido a que la escuela no se encuentra en estado ``Inscrita''.
 \end{UCtrayectoriaA}

 \begin{UCtrayectoriaA}[Fin del caso de uso]{B}{La fecha actual se encuentra fuera del periodo definido por la SMAGEM para el registro de escuelas}
    \UCpaso[\UCsist] Muestra el mensaje \cdtIdRef{MSG41}{Acción fuera del periodo} en la pantalla \cdtIdRef{IUR 9}{Administrar responsable del programa} indicando al actor que no puede modificar al responsable del programa debido a que la fecha actual se encuentra fuera del periodo definido por la SMAGEM para realizar la acción.
 \end{UCtrayectoriaA} 
 
 \begin{UCtrayectoriaA}[Fin del caso de uso]{C}{El actor desea cancelar la operación}
    \UCpaso[\UCactor] Solicita cancelar la operación oprimiendo el botón  \cdtButton{Cancelar} de la pantalla \cdtIdRef{IUR 11}{Modificar responsable del programa}
    \UCpaso[\UCsist] Regresa a la pantalla \cdtIdRef{IUR 9}{Administrar responsable del programa}
	
 \end{UCtrayectoriaA}
 
 \begin{UCtrayectoriaA}{D}{El actor no ingresa un dato que está marcado como requerido}
    \UCpaso[\UCsist] Muestra el mensaje \cdtIdRef{MSG5}{Falta un dato requerido para efectuar la operación solicitada} en la pantalla \cdtIdRef{IUR 11}{Modificar responsable del programa} indicando que se ha omitido un campo el cual es requerido. 
	    \UCpaso Continúa en el paso \ref{cur11:Acciones} de la trayectoria principal.
 \end{UCtrayectoriaA}
 
  \begin{UCtrayectoriaA}{E}{El actor ingresó un dato que no cumple con el formato requerido}
    \UCpaso[\UCsist] Muestra el mensaje \cdtIdRef{MSG6}{Formato incorrecto} en la pantalla \cdtIdRef{IUR 11}{Modificar responsable del programa} indicando que el valor ingresado es incorrecto ya que no cumple con el formato especificado.
    \UCpaso Continúa en el paso \ref{cur11:Acciones} de la trayectoria principal.
 \end{UCtrayectoriaA}
 
 
 \begin{UCtrayectoriaA}{F}{El actor ingresó un dato que no cumple con la longitud máxima requerida}
    \UCpaso[\UCsist] Muestra el mensaje \cdtIdRef{MSG7}{Se ha excedido la longitud máxima del campo} y señala el campo que presenta el tipo de dato erróneo en la pantalla \cdtIdRef{IUR 11}{Modificar responsable del programa}, indicando que la operación de registro no puede realizarse debido a que la información no es correcta. 
    \UCpaso Continúa en el paso \ref{cur11:Acciones} de la trayectoria principal.
 \end{UCtrayectoriaA}


 \begin{UCtrayectoriaA}{G}{El actor no proporciona un correo valido}
    \UCpaso[\UCsist] Muestra el mensaje \cdtIdRef{MSG16}{Error en formato de correo electrónico} en la pantalla \cdtIdRef{IUR 11}{Modificar responsable del programa} indicando que el correo electrónico proporcionado no cumple con un formato válido.
	    \UCpaso Continúa en el paso \ref{cur11:Acciones} de la trayectoria principal.
 \end{UCtrayectoriaA}
 
