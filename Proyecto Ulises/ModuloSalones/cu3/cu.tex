\begin{UseCase}{CUSM-01}{Consultar Asignación de Grupo}
    {
	Este caso de uso le permite al \cdtRef{actor:CIEAlumno}{Alumno} conocer el salón al que fue asignado el grupo en donde esta inscrito.
    }
    \UCitem{Versión}{1.0}
    \UCccsection{Administración de Requerimientos}
    \UCitem{Autor}{Ivo Sebastián Sam Álvarez-Tostado}
    \UCccitem{Evaluador}{José David Ortega Pacheco}
    \UCitem{Operación}{Consulta}
    \UCccitem{Prioridad}{Alta}
    \UCccitem{Complejidad}{Baja}
    \UCccitem{Volatilidad}{Baja}
    \UCccitem{Madurez}{Media}
    \UCitem{Estatus}{Por revisar}
    \UCitem{Fecha del último estatus}{15 de Octubre del 2017}


%--------------------------------------------------------
	\UCccsection{Revisión Versión 0.3} % Anote la versión que se revisó.
	% FECHA: Anote la fecha en que se terminó la revisión.
	\UCccitem{Fecha}{11-11-14} 
	% EVALUADOR: Coloque el nombre completo de quien realizó la revisión.
	\UCccitem{Evaluador}{Natalia Giselle Hernández Sánchez}
	% RESULTADO: Coloque la palabra que mas se apegue al tipo de acción que el analista debe realizar.
	\UCccitem{Resultado}{Corregir}
	% OBSERVACIONES: Liste los cambios que debe realizar el Analista.
	\UCccitem{Observaciones}{
		\begin{UClist}
			% PC: Petición de Cambio, describa el trabajo a realizar, si es posible indique la causa de la PC. Opcionalmente especifique la fecha en que considera razonable que se deba terminar la PC. No olvide que la numeración no se debe reiniciar en una segunda o tercera revisión.
			\RCitem{PC1}{\DONE{Agregar a precondiciones el estado de la cuenta}}{Fecha de entrega}
			\RCitem{PC2}{\DONE{Agregar el paso de la trayectoria de validación del estado de la cuenta}}{Fecha de entrega}
			\RCitem{PC3}{\DONE{Agregar el mensaje de cuenta no activada a la sección de errores}}{Fecha de entrega}
			\RCitem{PC4}{\DONE{Verificar las ligas a los estados}}{Fecha de entrega}
			
		\end{UClist}		
	}
%--------------------------------------------------------

	\UCsection{Atributos}
	\UCitem{Actor}{
		\begin{UClist} 
\UCli \cdtRef{actor:CIEAlumno}{Alumno}
	\end{UClist}
}
	\UCitem{Propósito}{Informar al alumno el salón al que fue asignado su grupo}
	\UCitem{Entradas}{
        \begin{UClist} 
           \UCli Ninguna
        \end{UClist}}
	\UCitem{Salidas}{
		\begin{UClist} 
			\UCli Tabla que muestra el listado de asignaciones de grupos
		\end{UClist}	
	}
	\UCitem{Precondiciones}{
		\begin{UClist}		
			\UCli {\bf Interna:} El sistema debe tener cargados las asignaciones de los grupos.
		\end{UClist}
		}
	\UCitem{Postcondiciones}{
	    \begin{UClist}
		\UCli {\bf Externa:} Los alumnos podrán saber en el momento que lo necesiten, el salón que fue asignado a su grupo sin tener que buscar las hojas de asignación de salones.
   	    \end{UClist}
	}
    \UCitem{Reglas de negocio}{
    	\begin{UClist}
%            \UCli \cdtIdRef{RN-S1}{Información correcta}: Verifica que la información introducida sea correcta.
	\end{UClist}
    }
	\UCitem{Errores}{
	    \begin{UClist}
%		\UCli \cdtIdRef{MSG5}{Falta un dato requerido para efectuar la operación solicitada}: Se muestra en la pantalla \cdtIdRef{IUR 1}{Iniciar sesión} cuando el actor omitió un dato marcado como requerido.
%		\UCli \cdtIdRef{MSG22}{Nombre de usuario y/o contraseña incorrecto}: Se muestra en la pantalla \cdtIdRef{IUR 1}{Iniciar sesión} indicando que el nombre de usuario y/o contraseña son incorrectos.
%		\UCli \cdtIdRef{MSG27}{Cuenta no activada}: Se muestra en la pantalla \cdtIdRef{IUR 1}{Iniciar sesión} indicando que la cuenta no está activada.
	    \end{UClist}
	}
	\UCitem{Tipo}{Primario.}
%	\UCitem{Fuente}{
 \end{UseCase}

 \begin{UCtrayectoria}
    \UCpaso[\UCactor] Desea conocer el salón ha donde fue asginado su grupo tocando el botón \botSalones de la pantalla \cdtIdRef{IUPP}{Pantalla Principal}
    
     \UCpaso[\UCsist] Obtiene la información de la asignación de grupos.
    
    \UCpaso[\UCsist] Muestra la pantalla \cdtIdRef{UISM-01}{Consultar Asignación de Grupos}
    
    \UCpaso[\UCsist] Muestra la tabla con los grupos y el nombre del salón al que fue asignado.
    
    \UCpaso[\UCactor] Desea conocer la ubicación deel nivel y salón al que fue asignado su grupo.
    
    \UCpaso[\UCsist] Ejecuta el caso de uso \cdtIdRef{CUSM-02}{Consultar Nivel de Salón}
    
 \end{UCtrayectoria}

\begin{UCtrayectoriaA}{A}{El alumno no desea conocer el nivel y salón.}
	\UCpaso[\UCactor] Toca el botón \cdtButton{Atrás} de la pantalla .
	\UCpaso[\UCsist] Muestra la pantalla \cdtIdRef{IUPP}{Pantalla Principal}.
\end{UCtrayectoriaA}
 
%\subsection{Puntos de extensión}
%
%\UCExtensionPoint
%{El actor requiere recuperar su contraseña}
%{ Paso \ref{cur1:Acciones} de la trayectoria principal}
%{\cdtIdRef{CUR 2}{Recuperar contraseña}}
%
%\UCExtensionPoint
%{El actor requiere solicitar la inscripción de su escuela al programa}
%{ Paso \ref{cur1:Acciones} de la trayectoria principal}
%{\cdtIdRef{CUR 3}{Solicitar inscripción}}
 