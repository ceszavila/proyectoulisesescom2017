
\begin{ReqSist}
	\reqSistItem{RF-001}{Edificios}{Para poder dar de alta los distintos salones, cubículos, salas de usos múltiples, etc., se cuenta con el número de edificio que lo identifica y debe ser el primer dígito del nombre o código del salón.}
    
    \reqSistItem{RF-002}{Niveles o pisos}{Así como los edificios, el nivel tiene un dígito identificador, es necesario obtener esa información para evitar falsos datos en el sistema. Dicho dígito conforma la segunda posición del nombre del espacio.}

	\reqSistItem{RF-003}{Tipo de espacio}{El sistema requiere saber si el espacios es un salón, laboratorio, sala de usos múltiples, sala de profesores, etc. De manera que éste puede tener un número y así conformar el código final del espacio que en caso de salón serían dos dígitos, así como también podría tener nombre de sala, número de sala, etc.}
    
    \reqSistItem{RF-004}{Orientación cardinal (únicamente norte y sur).}{En ocasiones el espacio es nombrado de acuerdo a su orientación cardinal, por lo que el sistema requiere contar con tal información para su correcto funcionamiento.}
    
    \reqSistItem{RF-005}{Profesores}{El sistema requiere tener el registro de la plantilla docente de la Escuela.}
    
    \reqSistItem{RF-006}{Ubicación de cubículos}{Para contar con la información sobre la ubicación de los profesores, el sistema requiere saber en qué cubículo o sala se encuentra cada profesor.}
    
	\reqSistItem{RF-007}{Información de profesores}{El sistema requiere contar con la academia a la que pertenece un profesor, horarios de atención, materias impartidas anteriores o actuales para su consulta y trabajos terminales dirigidos.}
    
   	\reqSistItem{RF-008}{Información personal y de contacto de profesores}{El sistema requiere contar con la información de personal como lo es la fotografía del profesor, así como los siguientes medios de contacto: cubículo, correo electrónico, página web.}
    
    \reqSistItem{RF-009}{Material de apoyo disponible actualmente.}{El sistema deberá tener el registro del material que actualmente utilizan los profesores y alumnos para su consulta.}
    
    \reqSistItem{RF-010}{Información del material de apoyo.}{Para el correcto funcionamiento del sistema y para facilitar las consultas del material, es necesario tener su información correcta, atributos como nombre, área de conocimiento, editoriales, etc.}
    
    \reqSistItem{RF-011}{Nombre del curso o certificación.}{El sistema podrá mostrar un registro de la información por orden alfabético o área de conocimiento.}
    
    \reqSistItem{RF-012}{Lugar y fecha de los cursos o certificaciones.}{Para ofrecer a los usuarios la información correcta, el sistema requiere saber la ubicación.}
    
     \reqSistItem{RF-013}{Ubicación del usuario.}{Para brindar a los usuarios la localización en tiempo real en un espacio de coordenadas, el sistema requiere saber la localización del usuario en un mapa.}
     
     \reqSistItem{RF-014}{Unidades de Aprendizaje.}{El sistema deberá tener registradas las unidades de aprendizaje que se impartirán.}
     
     \reqSistItem{RF-015}{Información de unidades de aprendizaje.}{Cada unidad de aprendizaje deberá tener la información necesaria para su consulta.}
     
     \reqSistItem{RF-016}{Programa académico de la unidad de aprendizaje.}{El sistema debe tener para la consulta, el programa académico de cada unidad de aprendizaje.}

	\reqSistItem{RF-017}{Convocatoria de movilidad.}{Será necesario obtener la convocatoria de movilidad cada vez que existan cambios en ella (se espera que sea cada seis meses).}
\end{ReqSist}