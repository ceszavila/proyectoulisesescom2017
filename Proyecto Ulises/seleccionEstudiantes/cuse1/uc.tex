
\begin{UseCase}{CUSE 1}{Generar listas de aspirantes}
    {
    	El \textbf{SAEV2.0} genera la lista de aspirantes de aceptados y la lista de aspirantes de no aceptados que concluyeron con el proceso de admisión para mostrar esta información a la \textbf{Coordinación de Control Escolar} o \textbf{Secretaria de Administración}.
    }

    \UCitem{Versión}{1.0}
    \UCccsection{Administración de Requerimientos}
    \UCitem{Autor}{Diego Efrén Pascual Hernández}
    \UCccitem{Evaluador}{}
    \UCitem{Operación}{Generar}
    \UCccitem{Prioridad}{Alta}
    \UCccitem{Complejidad}{Baja}
    \UCccitem{Volatilidad}{}
    \UCccitem{Madurez}{}
    \UCitem{Estatus}{Terminado}
    \UCitem{Fecha del último estatus}{17 marzo 2017}
    
%% Copie y pegue este bloque tantas veces como revisiones tenga el caso de uso.
%% Esta sección la debe llenar solo el Revisor
% %--------------------------------------------------------
 	\UCccsection{Revisión Versión 1.0} % Anote la versión que se revisó.
% 	% FECHA: Anote la fecha en que se terminó la revisión.
 	\UCccitem{Fecha}{9-Dic} 
% 	% EVALUADOR: Coloque el nombre completo de quien realizó la revisión.
 	\UCccitem{Evaluador}{Nayeli Vega}
% 	% RESULTADO: Coloque la palabra que mas se apegue al tipo de acción que el analista debe realizar.
 	\UCccitem{Resultado}{Corregir}
% 	% OBSERVACIONES: Liste los cambios que debe realizar el Analista.
 	\UCccitem{Observaciones}{
 		\begin{UClist}
% 			% PC: Petición de Cambio, describa el trabajo a realizar, si es posible indique la causa de la PC. Opcionalmente especifique la fecha en que considera razonable que se deba terminar la PC. No olvide que la numeración no se debe reiniciar en una segunda o tercera revisión.
 			\RCitem{PC1}{\TODO{Resumen: sugiero ... objetivo de la línea de acción de ambiente escolar.}}{Fecha de entrega}
 			\RCitem{PC2}{\TODO{Modificar propósito }}{Fecha de entrega} 			
			\RCitem{PC3}{\TODO{Agregar precondición de estado. Que la escuela se encuentre en estado Avance en edición }}{Fecha de entrega} 					 			
			\RCitem{PC4}{\TODO{Agregar precondición de fecha }}{Fecha de entrega} 					 				
			\RCitem{PC5}{\TODO{Agregar mensaje 41 en los errores}}{Fecha de entrega} 					 							\RCitem{PC6}{\TODO{Agregar a la trayectria las dos verificaciones, estado y fecha}}{Fecha de entrega} 	
			\RCitem{PC7}{\TODO{Agregar rutas alternativas }}{Fecha de entrega} 					 					
			\RCitem{PC8}{\TODO{Interfaz: agregar mensaje 41 }}{Fecha de entrega} 									 									
% 			\RCitem{PC2}{\TODO{Descripción del pendiente}}{Fecha de entrega}
% 			\RCitem{PC3}{\TODO{Descripción del pendiente}}{Fecha de entrega}
 		\end{UClist}		
 	}
% %--------------------------------------------------------

    \UCsection{Atributos}
    \UCitem{Actor(es)}{\textbf{SAEV2.0}}
    \UCitem{Propósito}{Generar listas de aspirantes (Aceptados y No aceptados) que concluyeron con el proceso de admisión}
    \UCitem{Entradas}{Ninguna.}
    \UCitem{Salidas}{
    	\begin{UClist}
    		\UCli Lista de aspirantes aceptados.
    		\UCli Lista de aspirantes no aceptados.
    	\end{UClist}
    }
    \UCitem{Precondiciones}{
	\begin{UClist}
	    \UCli {\bf Interna:} Que haya concluido el periodo de entrevistas.
        \UCli {\bf Interna:} Que la \textbf{Coordinación de Control Escolar} o \textbf{Secretaria de Administración} haya solicitado ver los resultados de los aspirantes.
	\end{UClist}
    }
    
    \UCitem{Postcondiciones}{
	\begin{UClist}
	    \UCli {\bf Interna:} Se podrán visualizar las listas de aspirantes aceptados y de no aceptados.
	\end{UClist}
    }

    %Reglas de negocio: Especifique las reglas de negocio que utiliza este caso de uso
    \UCitem{Reglas de negocio}{Ninguna.}
    \UCitem{Errores}{
    \begin{UClist}
        \UCli \cdtIdRef{MSG28}{Operación no permitida por estado de la entidad}: Se muestra en la pantalla \cdtIdRef{IUS 1}{Administrar avances de objetivos} indicando al actor que no se puede administrar los avances de meta ya que el Plan de acción no se encuentra aprobado.
    \end{UClist}
    }
    \UCitem{Tipo}{Secundario, incluye en el caso de uso \cdtIdRef{CUSE 2}{Visualizar lista de aspirantes aceptados}.}
\end{UseCase}



 \begin{UCtrayectoria}
    \UCpaso[\UCactor] Oprime el botón \botAcciones del objetivo que desea administrar en la pantalla \cdtIdRef{IUS 1}{Administrar avances de objetivos}.
    \label{cus2:oprimeAvance}
    \UCpaso[\UCsist] Verifica que la escuela se encuentre en estado ``Avance en edición''. \refTray{A}.
    \UCpaso[\UCsist] Verifica que la fecha actual se encuentre dentro del periodo definido por la SMAGEM para registrar avances en las acciones de agua. \refTray{B}.
     \UCpaso[\UCsist] Muestra la información de las metas de ``Ambiente escolar'' en la pantalla \cdtIdRef{IUS 2}{Administrar avances de ambiente escolar}.
     \UCpaso[\UCactor] Administra los avances de meta o acciones a través de los botones \botAutoAjus y \botMetas. \label{cus2:Administrar}
    \end{UCtrayectoria}

    \begin{UCtrayectoriaA}[Fin del caso de uso]{A}{La escuela no se encuentra en un estado que permita registrar avance en las acciones}
    \UCpaso[\UCactor] Muestra el mensaje \cdtIdRef{MSG28}{Operación no permitida por estado de la entidad} en la pantalla \cdtIdRef{IUS 1}{Administrar avances de objetivos} indicando al actor que no se puede administrar los avances por el estado en que se encuentra la escuela.
    \end{UCtrayectoriaA}

    \begin{UCtrayectoriaA}[Fin del caso de uso]{B}{La fecha actual se encuentra fuera del periodo definido por la SMAGEM para el registro de avances.}
    \UCpaso[\UCsist] Muestra el mensaje \cdtIdRef{MSG41}{Acción fuera del periodo} en la pantalla \cdtIdRef{IUS 1}{Administrar avances de objetivos} indicando al actor que no puede registrar avances debido a que la fecha actual se encuentra fuera del periodo definido por la SMAGEM para realizar la acción. 
    \end{UCtrayectoriaA}

\subsection{Puntos de extensión}

\UCExtensionPoint
{El Usuario desea registrar un avance de meta}
{Paso \ref{cus2:Administrar} de la Trayectoria Principal}
{\cdtRef{CUS 3}{Registrar avance de meta de ambiente escolar}}

\UCExtensionPoint
{El Usuario desea registrar un avance de las acciones de una meta}
{Paso \ref{cus2:Administrar} de la Trayectoria Principal}
{\cdtRef{CUS 2}{Registrar avance de acciones de ambiente escolar}}
