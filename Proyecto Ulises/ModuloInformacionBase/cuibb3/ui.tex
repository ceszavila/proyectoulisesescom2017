\subsection{IUIBB 3 Administrar inventario de fauna}

\subsubsection{Objetivo}

    En esta pantalla el \cdtRef{actor:usuarioEscuela}{Coordinador del programa} puede conocer las especies animales que se encuentran ubicadas en los ecosistemas y áreas verdes cercanas a la escuela, el número total de especies y el número total de especies endémicas, además puede acceder a la eliminación de registros de especies.

\subsubsection{Diseño}

    En la figura~\ref{IUIBB 3} se muestra la pantalla ``Administrar inventario de fauna'', por medio de la cual se podrá acceder al registro, consulta y eliminación de registros de especies animales.\\
    
    El actor tendrá la facultad de consultar las especies animales que se encuentran en los ecosistemas y áreas verdes cercanas a la escuela, registrar nuevas especies animales o eliminarlas, esto a través de los botones \cdtButton{Registrar} y \botKo.\\
    
    En el caso de que no existan registros de especies animales en el sistema los campos de ``Total de especies'' y ``Total de especies endémicas'' aparecerán con el número ``0''. Además se mostrará el mensaje \cdtIdRef{MSG2}{No existe información registrada por el momento} indicando que no se encuentran registros de especies animales en el sistema.

  \IUfig[.9]{pantallas/InformacionBase/cuibb3/IUIBB3AdministrarFauna.png}{IUIBB 3}{Administrar inventario de fauna}


\subsubsection{Comandos}
    \begin{itemize}
	\item \cdtButton{Registrar}: Permite al actor registrar especies animales en el sistema, dirige a la pantalla \cdtIdRef{IUIBB 4}{Registrar fauna}.
	
	\item \botKo[Eliminar]: Permite al actor eliminar especies animales del sistema, dirige a la pantalla emergente \cdtIdRef{IUIBB 5}{Eliminar fauna}.
	
	\item \cdtButton{Regresar}: Permite al actor cancelar la administración de inventario de fauna, dirige a la pantalla \cdtIdRef{IUIBB 1}{Administrar información base para indicadores de biodiversidad}.
    \end{itemize}

\subsubsection{Mensajes}

    \begin{description}
	\item [\cdtIdRef{MSG2}{No existe información registrada por el momento}:] Se muestra en la pantalla \cdtIdRef{IUIBB 3}{Administrar inventario de fauna} indicando al actor que no existen registros de especies animales en el sistema por el momento.
	
	\item[\cdtIdRef{MSG28}{Operación no permitida por estado de la entidad}:] Se muestra en la pantalla \cdtIdRef{IUIBB 1}{Administrar información base para indicadores de biodiversidad} indicando al actor que no puede realizar la operación debido al estado en que se encuentra la escuela.
	
	\item [\cdtIdRef{MSG41}{Acción fuera del periodo}:] Se muestra sobre la pantalla en la pantalla \cdtIdRef{IUIBB 1}{Administrar información base para indicadores de biodiversidad} para indicarle al actor que no puede realizar la operación debido a que la fecha actual se encuentra fuera del periodo definido por la SMAGEM para realizarla.
    \end{description}