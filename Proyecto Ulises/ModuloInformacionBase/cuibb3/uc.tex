%!TEX encoding = UTF-8 Unicode

\begin{UseCase}{CUIBB 3}{Administrar inventario de fauna}
    {
	La fauna es el conjunto de especies animales que habitan en una región. Este caso de uso permite al actor conocer las especies que se encuentran ubicadas en los ecosistemas y áreas verdes cercanas a la escuela, el número total de especies y el número total de especies endémicas, además sirve como punto de acceso para registrar y eliminar especies. 
    }
    
    \UCitem{Versión}{1.0}
    \UCccsection{Administración de Requerimientos}
    \UCitem{Autor}{Jessica Stephany Becerril Delgado}
    \UCccitem{Evaluador}{}
    \UCitem{Operación}{Administración}
    \UCccitem{Prioridad}{Media}
    \UCccitem{Complejidad}{Media}
    \UCccitem{Volatilidad}{Alta}
    \UCccitem{Madurez}{Media}
    \UCitem{Estatus}{Terminado}
    \UCitem{Fecha del último estatus}{20 de Noviembre del 2014}
    
%% Copie y pegue este bloque tantas veces como revisiones tenga el caso de uso.
%% Esta sección la debe llenar solo el Revisor
% %--------------------------------------------------------
% 	\UCccsection{Revisión Versión XX} % Anote la versión que se revisó.
% 	% FECHA: Anote la fecha en que se terminó la revisión.
% 	\UCccitem{Fecha}{Fecha en que se termino la revisión} 
% 	% EVALUADOR: Coloque el nombre completo de quien realizó la revisión.
% 	\UCccitem{Evaluador}{Nombre de quien revisó}
% 	% RESULTADO: Coloque la palabra que mas se apegue al tipo de acción que el analista debe realizar.
% 	\UCccitem{Resultado}{Corregir, Desechar, Rehacer todo, terminar.}
% 	% OBSERVACIONES: Liste los cambios que debe realizar el Analista.
% 	\UCccitem{Observaciones}{
% 		\begin{UClist}
% 			% PC: Petición de Cambio, describa el trabajo a realizar, si es posible indique la causa de la PC. Opcionalmente especifique la fecha en que considera razonable que se deba terminar la PC. No olvide que la numeración no se debe reiniciar en una segunda o tercera revisión.
% 			\RCitem{PC1}{\TODO{Descripción del pendiente}}{Fecha de entrega}
% 			\RCitem{PC2}{\TODO{Descripción del pendiente}}{Fecha de entrega}
% 			\RCitem{PC3}{\TODO{Descripción del pendiente}}{Fecha de entrega}
% 		\end{UClist}		
% 	}
% %--------------------------------------------------------

    \UCsection{Atributos}
    \UCitem{Actor}{\cdtRef{actor:usuarioEscuela}{Coordinador del programa}}
    \UCitem{Propósito}{Administrar el registro y eliminación de las especies animales que se encuentran en los ecosistemas y áreas verdes cercanas a la escuela.}
    \UCitem{Entradas}{
	\begin{UClist}
	    \UCli Ninguna.
	\end{UClist}
    }
    \UCitem{Salidas}{
	\begin{UClist} 
	    \UCli \cdtRef{inventario:cantidad}{Total de especies}: \ioCalcular \cdtIdRef{RN-N10}{Calcular total de especies}.
	    \UCli \cdtRef{gls:endemico}{Total de especies endémicas}: \ioCalcular \cdtIdRef {RN-N11}{Calcular total de especies endémicas}.
	    \UCli \cdtRef{inventario}{Fauna}: \ioTabla{\cdtRef{gls:categoriaFauna}{Categoría}, \cdtRef{inventario:nombreComun}{Nombre común especie}, \cdtRef{inventario:nombreCientifico}{Nombre científico}, \cdtRef{gls:endemico}{Endémica}, \cdtRef{gls:riesgo}{En riesgo de desaparecer}, \cdtRef{inventario:cantidad}{Cantidad} y  \cdtRef{inventario:ubicacion}{Ubicación}}{que estén en el sistema}.
	\end{UClist}
    }

    \UCitem{Precondiciones}{
	\begin{UClist}
	    \UCli {\bf Interna:} Que la escuela se encuentre en estado \cdtRef{estado:infoBaseEdicion}{Información base en edición}.
	    \UCli {\bf Interna:} Que el periodo de registro de información base se encuentre vigente. 
	\end{UClist}
    }
    
    \UCitem{Postcondiciones}{
	\begin{UClist}
	    \UCli {\bf Interna:} Se podrá registrar una especie animal a través del caso de uso \cdtIdRef{CUIBB 4}{Registrar fauna}.
	    \UCli {\bf Interna:} Se podrá eliminar una especie animal a través del caso de uso \cdtIdRef{CUIBB 5}{Eliminar fauna}.
	\end{UClist}
    }
    
    \UCitem{Reglas de negocio}{
    	\begin{UClist}
	    \UCli \cdtIdRef{RN-N10}{Calcular total de especies}: Calcula el número total de especies registradas para el inventario de fauna.
	    \UCli \cdtIdRef{RN-N11}{Calcular total de especies endémicas}: Calcula el número total de especies endémicas registradas para el inventario de fauna.
	\end{UClist}
    }
    
    \UCitem{Errores}{
	\begin{UClist}
	    \UCli \cdtIdRef{MSG2}{No existe información registrada por el momento}: Se muestra en la pantalla \cdtIdRef{IUIBB 3}{Administrar inventario de fauna} indicando al actor que no existen registros de especies animales en el sistema por el momento.
	    
	    \UCli \cdtIdRef{MSG28}{Operación no permitida por estado de la entidad}: Se muestra en la pantalla \cdtIdRef{IUIBB 1}{Administrar información base para indicadores de biodiversidad} indicando al actor que no se puede realizar la operación debido al estado en que se encuentra la escuela.
	    
	    \UCli \cdtIdRef{MSG41}{Acción fuera del periodo}: Se muestra sobre la pantalla en la pantalla \cdtIdRef{IUIBB 1}{Administrar información base para indicadores de biodiversidad} para indicarle al actor que no puede realizar la operación debido a que la fecha actual se encuentra fuera del periodo definido por la SMAGEM para realizarla.
	    
	\end{UClist}
    }

    \UCitem{Tipo}{Secundario, extiende del caso de uso \cdtIdRef{CUIBB 1}{Administrar información base para indicadores de biodiversidad}.}

%    \UCitem{Fuente}{
%	\begin{UClist}
%	    \UCli Minuta de la reunión \cdtIdRef{M-17RT}{Reunión de trabajo}.
%	\end{UClist}
 %   }
\end{UseCase}

 \begin{UCtrayectoria}
    \UCpaso[\UCactor] Solicita administrar el inventario de fauna oprimiendo el botón \botReg en la pantalla \cdtIdRef{IUIBB 1}{Administrar información base para indicadores de biodiversidad}.
    \UCpaso[\UCsist] Verifica que la escuela se encuentre en estado ``Información base en edición''. \refTray{A}.
    \UCpaso[\UCsist] Verifica que la fecha actual se encuentre dentro del periodo definido por la SMAGEM para realizar la operación. \refTray{B}.
    \UCpaso[\UCsist] Busca la información de las especies animales registradas en el sistema. \refTray{C}.
    \UCpaso[\UCsist] Muestra la pantalla \cdtIdRef{IUIBB 3}{Administrar inventario de fauna}.
    \UCpaso[\UCactor] Administra el inventario de fauna a través de los botones \cdtButton{Registrar}y \botKo  . \label{cuibb3:Registrar}
 \end{UCtrayectoria}
 
 \begin{UCtrayectoriaA}[Fin del caso de uso]{A}{La escuela no se encuentra en un estado que permita realizar la operación.}
    \UCpaso[\UCsist] Muestra el mensaje \cdtIdRef{MSG28}{Operación no permitida por estado de la entidad} en la pantalla \cdtIdRef{IUIBB 1}{Administrar información base para indicadores de biodiversidad} indicando al actor que no puede realizar la operación debido al estado en que se encuentra la escuela.
 \end{UCtrayectoriaA}
 
    \begin{UCtrayectoriaA}[Fin del caso de uso]{B}{La fecha actual se encuentra fuera del periodo definido por la SMAGEM para realizar la operación.}
    \UCpaso[\UCsist] Muestra el mensaje \cdtIdRef{MSG41}{Acción fuera del periodo} en la pantalla \cdtIdRef{IUIBB 1}{Administrar información base para indicadores de biodiversidad} indicando al actor que no puede realizar la operación debido a que la fecha actual se encuentra fuera del periodo definido por la SMAGEM para realizar la acción. 
 \end{UCtrayectoriaA}
 
  \begin{UCtrayectoriaA}[Fin del caso de uso]{C}{No hay registros de especies animales para mostrar.}
    \UCpaso[\UCsist] Muestra el mensaje \cdtIdRef{MSG2}{No existe información registrada por el momento} en la pantalla \cdtIdRef{IUIBB 3}{Administrar inventario de fauna} indicando al actor que aún no hay especies animales registradas. 
 \end{UCtrayectoriaA}


\subsection{Puntos de extensión}

\UCExtensionPoint
{El actor desea registrar una especie animal}
{ Paso \ref{cuibb3:Registrar} de la trayectoria principal}
{\cdtIdRef{CUIBB 4}{Registrar fauna}}

\UCExtensionPoint
{El actor desea eliminar una especie animal}
{ Paso \ref{cuibb3:Registrar} de la trayectoria principal}
{\cdtIdRef{CUIBB 5}{Eliminar fauna}}
