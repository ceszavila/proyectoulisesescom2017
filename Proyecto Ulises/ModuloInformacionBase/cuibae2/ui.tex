\subsection{IUIBAE 2 Registrar información base para indicadores de ambiente escolar}

\subsubsection{Objetivo}

      En esta pantalla el \cdtRef{actor:usuarioEscuela}{Coordinador del programa} puede registrar o modificar la información base para indicadores de ambiente escolar que permita conocer el estado en que se encuentra una escuela con respecto a sus instalaciones, espacios de estudio y recreación.

\subsubsection{Diseño}

    En la figura~\ref{IUIBAE 2} se muestra la pantalla ``Registrar información base para indicadores de ambiente escolar'', por medio de la cual se podrá registrar la información base que permitirá conocer el estado en que se encuentra una escuela con respecto a sus instalaciones, espacios de estudio y recreación. El actor deberá ingresar la información solicitada en la pantalla, en el caso de que se trate de una modificación de la información aparecerán los datos previamente ingresados en los campos respectivos de la pantalla.\\
    
    Cuando se seleccione alguna de las opciones para la pregunta ``Seleccione los tipos de espacios con los que cuenta la escuela'' se mostrará debajo de la opción un cuadro de texto en el cual deberá ingresar una descripción del uso que se le da a este tipo de espacio en la escuela, como se muestra en la figura~\ref{IUIBAE 2.3}.\\ 
    
    Una vez que se haya ingresado toda la información solicitada para el registro de la información deberá oprimir el botón \cdtButton{Aceptar}, si se trata de una modificación se mostrará el mensaje \cdtIdRef{MSG30}{Confirmar la modificación de un registro} en una pantalla emergente como se muestra en la figura~\ref{IUIBAE 2.1} para indicar al actor que la información previa a la modificación se perderá. El sistema validará y registrará la información sólo si se han cumplido todas las reglas de negocio establecidas.\\
    
    Finalmente se mostrará el mensaje \cdtIdRef{MSG1}{Operación realizada exitosamente} en la pantalla \cdtIdRef{IUIBAE 1}{ Administrar información base para indicadores de ambiente escolar}, para indicar que la información base se ha registrado o modificado exitosamente.
      
    \IUfig[.9]{pantallas/InformacionBase/cuibae2/IUIBAE2RegistrarInformacion.png}{IUIBAE 2}{Registrar información base para indicadores de ambiente escolar}
    \IUfig[.7]{pantallas/InformacionBase/cuibae2/IUIBAE2ModificarInformacion.png}{IUIBAE 2.1}{Registrar información base para indicadores de ambiente escolar: Mensaje de confirmación}
    \IUfig[.9]{pantallas/InformacionBase/cuibae2/IUIBAE2RegistrarInformacion1.png}{IUIBAE 2.3}{Registrar información base para indicadores de ambiente escolar: Uso del espacio}

\subsubsection{Comandos}
    \begin{itemize}
	\item \cdtButton{Aceptar}: Permite al actor confirmar el registro o modificación de la información base para indicadores de ambiente escolar, dirige a la pantalla \cdtIdRef{IUIBAE 1}{ Administrar información base para indicadores de ambiente escolar}.
	\item \cdtButton{Cancelar}: Permite al actor cancelar el registro o modificación de la información base para indicadores de ambiente escolar, dirige a la pantalla \cdtIdRef{IUIBAE 1}{ Administrar información base para indicadores de ambiente escolar}.
    \end{itemize}

\subsubsection{Mensajes}

    \begin{description}
      
	    \item [\cdtIdRef{MSG1}{Operación realizada exitosamente}:] Se muestra en la pantalla \cdtIdRef{IUIBAE 1}{Administrar información base para indicadores de ambiente escolar} cuando el registro de la información base para indicadores de ambiente escolar se ha realizado correctamente.
	    
	    \item [\cdtIdRef{MSG5}{Falta un dato requerido para efectuar la operación solicitada}:] Se muestra en la pantalla \cdtIdRef{IUIBAE 2}{Registrar información base para indicadores de ambiente escolar} cuando no se ha ingresado un dato marcado como requerido.
	    
	    \item [\cdtIdRef{MSG6}{Formato incorrecto}:] Se muestra en la pantalla \cdtIdRef{IUIBAE 2}{Registrar información base para indicadores de ambiente escolar} cuando el tipo de dato ingresado no cumple con el tipo de dato solicitado en el campo.
	    
	    \item [\cdtIdRef{MSG7}{Se ha excedido la longitud máxima del campo}:] Se muestra en la pantalla \cdtIdRef{IUIBAE 2}{Registrar información base para indicadores de ambiente escolar} cuando se ha excedido la longitud de alguno de los campos.	    
	    
	    \item[\cdtIdRef{MSG28}{Operación no permitida por estado de la entidad}:] Se muestra en la pantalla \cdtIdRef{IUIBAE 1}{Administrar información base para indicadores de ambiente escolar} indicando al actor que no se puede realizar la operación debido al estado en que se encuentra la escuela.
	    
	    \item [\cdtIdRef{MSG30}{Confirmar la modificación de un registro}:] Se muestra en la pantalla emergente \cdtIdRef{IUIBAE 2.1}{Registrar información base para indicadores de ambiente escolar: Mensaje de confirmación} para indicar al actor que al guardar los cambios realizados en la información base la información previa se perderá.
	    
	    \item [\cdtIdRef{MSG41}{Acción fuera del periodo}:] Se muestra sobre la pantalla en la pantalla \cdtIdRef{IUIBAE 1}{Administrar información base para indicadores de ambiente escolar} para indicarle al actor que no puede realizar la operación debido a que la fecha actual se encuentra fuera del periodo definido por la SMAGEM para realizarla.
    \end{description}
