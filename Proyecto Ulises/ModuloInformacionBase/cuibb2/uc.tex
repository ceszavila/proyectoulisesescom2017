%!TEX encoding = UTF-8 Unicode

\begin{UseCase}{CUIBB 2}{Registrar información base para indicadores de biodiversidad}
    {
	  Este caso de uso permite al actor el registro o modificación de informacón base que contribuya como referencia para los indicadores de biodiversidad, proporcionando una visión general del número de personas que conocen el concepto de biodiversidad en el ámbito escolar, además de conocer la información referente a las áreas verdes y ecosistemas cercanos a la escuela. Una vez que los datos solicitados para el registro de la información base para indicadores de biodiversidad han sido ingresados, el sistema valida la información y esta queda registrada.
    }
    
    \UCitem{Versión}{1.0}
    \UCccsection{Administración de Requerimientos}
    \UCitem{Autor}{Jessica Stephany Becerril Delgado}
    \UCccitem{Evaluador}{}
    \UCitem{Operación}{Registro}
    \UCccitem{Prioridad}{Media}
    \UCccitem{Complejidad}{Media}
    \UCccitem{Volatilidad}{Media}
    \UCccitem{Madurez}{Media}
    \UCitem{Estatus}{Terminado}
    \UCitem{Fecha del último estatus}{20 de Noviembre del 2014}

    \UCsection{Atributos}
    \UCitem{Actor}{\cdtRef{actor:usuarioEscuela}{Coordinador del programa}}
    \UCitem{Propósito}{Registrar o modificar la información referente a las áreas verdes y ecosistemas cercanos a la escuela, así como también al número de personas que entienden el concepto de biodiversidad dentro del ámbito escolar.}
    \UCitem{Entradas}{
	\begin{UClist}
	    \UCli Información de biodiversidad.
	    \begin{itemize}
	    \item \cdtRef{biodiversidad:encuestados}{Número de personas encuestadas}: \ioEscribir.
	    \item \cdtRef{biodiversidad:siConocenConcepto}{Número de personas que conocen el concepto de biodiversidad}: \ioEscribir. 
	    \end{itemize}
\end{UClist}
    }
    \UCitem{Entradas}{
	\begin{UClist}
	
	    \UCli Información de los ecosistemas.
	    \begin{itemize}
	     \item \cdtRef{biodiversidad:tipoDeEcosistema}{Tipos de ecosistemas}: \ioCheckBox.
	     \item \cdtRef{biodiversidad:distanciaRio}{Distancia desde la escuela al río}: \ioEscribir.
	     \item \cdtRef{biodiversidad:ubicacionRio}{Ubicación}: \ioEscribir.
	     \item \cdtRef{biodiversidad:distanciaBosque}{Distancia desde la escuela al bosque}: \ioEscribir.
	     \item \cdtRef{biodiversidad:ubicacionBosque}{Ubicación del bosque}: \ioEscribir.
	     \item \cdtRef{biodiversidad:distanciaSelva}{Distancia desde la escuela a la selva}: \ioEscribir.
	     \item \cdtRef{biodiversidad:ubicacionSelva}{Ubicación de la selva}: \ioEscribir.
	     \item \cdtRef{biodiversidad:distanciaMatorral}{Distancia desde la escuela al matorral}: \ioEscribir.
	     \item \cdtRef{biodiversidad:ubicacionMatorral}{Ubicación del matorral}: \ioEscribir.
	     \item \cdtRef{biodiversidad:distanciaEstanque}{Distancia desde la escuela al estanque}: \ioEscribir.
	     \item \cdtRef{biodiversidad:ubicacionEstanque}{Ubiación del estanque}: \ioEscribir.
	    \end{itemize}

	    \UCli Información de las áreas verdes
	    \begin{itemize}
	     \item \cdtRef{biodiversidad:superficieAreasVerdes}{Superficie del predio de áreas verdes}: \ioEscribir.
	     \item \cdtRef{biodiversidad:tipoAreaVerde}{Tipos de áreas verdes}: \ioCheckBox.
	    \end{itemize}
	\end{UClist}
    }

    \UCitem{Salidas}{
	\begin{UClist} 
	    	    \UCli Información de biodiversidad.
	    \begin{itemize}
	    \item \cdtRef{biodiversidad:encuestados}{Número de personas encuestadas}: \ioObtener.
	    \item \cdtRef{biodiversidad:siConocenConcepto}{Número de personas que conocen el concepto de biodiversidad}: \ioObtener. 
	    \end{itemize}
	    \UCli Información de los ecosistemas.
	    \begin{itemize}
	     \item \cdtRef{biodiversidad:tipoDeEcosistema}{Tipos de ecosistemas}: \ioObtener.
	     \item \cdtRef{biodiversidad:distanciaRio}{Distancia desde la escuela al río}: \ioObtener.
	     \item \cdtRef{biodiversidad:ubicacionRio}{Ubicación del río}: \ioObtener.
	     \item \cdtRef{biodiversidad:distanciaBosque}{Distancia desde la escuela al bosque}: \ioObtener.
	     \item \cdtRef{biodiversidad:ubicacionBosque}{Ubicación del bosque}: \ioObtener.
	     \item \cdtRef{biodiversidad:distanciaSelva}{Distancia desde la escuela a la selva}: \ioObtener.
	     \item \cdtRef{biodiversidad:ubicacionSelva}{Ubicación de la selva}: \ioObtener.
	     \item \cdtRef{biodiversidad:distanciaMatorral}{Distancia desde la escuela al matorral}: \ioObtener.
	     \item \cdtRef{biodiversidad:ubicacionMatorral}{Ubicación del matorral}: \ioObtener.
	     \item \cdtRef{biodiversidad:distanciaEstanque}{Distancia desde la escuela al estanque}: \ioObtener.
	     \item \cdtRef{biodiversidad:ubicacionEstanque}{Ubiación del estanque}: \ioObtener.
	    \end{itemize}

	    \UCli Información de las áreas verdes
	    \begin{itemize}
	     \item \cdtRef{escuela:superficieTotal}{Superficie del predio total}: \ioObtener.
	     \item \cdtRef{escuela:superficieConstruida}{Superficie del predio construido}: \ioObtener.
	     \item \cdtRef{biodiversidad:superficieAreasVerdes}{Superficie del predio de áreas verdes}: \ioObtener.
	     \item \cdtRef{biodiversidad:tipoAreaVerde}{Tipos de áreas verdes}: \ioObtener.
	    \end{itemize}   
	    \UCli \cdtIdRef{MSG1}{Operación realizada exitosamente:} Se muestra en la pantalla \cdtIdRef{IUIBB 1}{Administrar información base para indicadores de biodiversidad} cuando el registro de la información base para indicadores de biodiversidad se ha realizado correctamente.
	    
	    \UCli \cdtIdRef{MSG30}{Confirmar la modificación de un registro}: Se muestra en la pantalla emergente \cdtIdRef{IUIBB 2.1}{Registrar información base para indicadores de biodiversidad: Mensaje de confirmación} para indicar al actor que al guardar los cambios realizados la información previa se perderá.  
	\end{UClist}
    }

    \UCitem{Precondiciones}{
	\begin{UClist}
	    \UCli {\bf Interna:} Que la escuela se encuentre en estado \cdtRef{estado:infoBaseEdicion}{Información base en edición}.
	    \UCli {\bf Interna:} Que el periodo de registro de información base se encuentre vigente. 
	\end{UClist}
    }
    
    \UCitem{Postcondiciones}{
	\begin{UClist}
	    \UCli {\bf Interna:} Se podrá modificar la información base definida para los indicadores de biodiversidad a través del caso de uso \cdtIdRef{CUIBB 2}{Registrar información base para indicadores de biodiversidad}.
	\end{UClist}
    }
    
    \UCitem{Reglas de negocio}{
    	\begin{UClist}
	    \UCli \cdtIdRef{RN-S1}{Información correcta}: Verifica que la información introducida sea correcta.
	\end{UClist}
    }
    
    \UCitem{Errores}{
	\begin{UClist}
	    
	    \UCli \cdtIdRef{MSG5}{Falta un dato requerido para efectuar la operación solicitada}: Se muestra en la pantalla \cdtIdRef{IUIBB 2}{Registrar información base para indicadores de biodiversidad} cuando no se ha ingresado un dato marcado como requerido.
	    
	     \UCli \cdtIdRef{MSG6}{Formato incorrecto}: Se muestra en la pantalla \cdtIdRef{IUIBB 2}{Registrar información base para indicadores de biodiversidad} cuando el tipo de dato ingresado no cumple con el tipo de dato solicitado en el campo.
	    
	    \UCli \cdtIdRef{MSG7}{Se ha excedido la longitud máxima del campo}: Se muestra en la pantalla \cdtIdRef{IUIBB 2}{Registrar información base para indicadores de biodiversidad} cuando se ha excedido la longitud de alguno de los campos.	    
	    
	    \UCli \cdtIdRef{MSG28}{Operación no permitida por estado de la entidad}: Se muestra en la pantalla \cdtIdRef{IUIBB 1}{Administrar información base para indicadores de biodiversidad} indicando al actor que no se puede realizar la operación debido al estado en que se encuentra la escuela.
	    
	    \UCli \cdtIdRef{MSG41}{Acción fuera del periodo}: Se muestra sobre la pantalla en la pantalla \cdtIdRef{IUIBB 1}{Administrar información base para indicadores de biodiversidad} para indicarle al actor que no puede realizar la operación debido a que la fecha actual se encuentra fuera del periodo definido por la SMAGEM para realizarla.
	\end{UClist}
    }

    \UCitem{Tipo}{Secundario, extiende del caso de uso \cdtIdRef{CUIBB 1}{Administrar información base para indicadores de biodiversidad}.}

%    \UCitem{Fuente}{
%	\begin{UClist}
%	    \UCli Minuta de la reunión \cdtIdRef{M-17RT}{Reunión de trabajo}.
%	\end{UClist}
%    }
\end{UseCase}

\begin{UCtrayectoria}
    \UCpaso[\UCactor] Solicita registrar la información base para indicadores de biodiversidad oprimiendo el botón \botReg de la pantalla \cdtIdRef{IUIBB 1}{Administrar información base para indicadores de biodiversidad}.
    \UCpaso[\UCsist] Verifica que la escuela se encuentre en estado ``Información base en edición''. \refTray{A}.
    \UCpaso[\UCsist] Verifica que la fecha actual se encuentre dentro del periodo definido por la SMAGEM para realizar la operación. \refTray{B}.
    \UCpaso[\UCsist] Busca la información previamente registrada referente a la información base para indicadores de biodiversidad.
    \UCpaso[\UCsist] Muestra la pantalla \cdtIdRef{IUIBB 2}{Registrar información base para indicadores de biodiversidad} por medio de la cual se realizará el registro de información base para indicadores de biodiversidad.
    \UCpaso[\UCactor] Ingresa la información referente a la información base para indicadores de biodiversidad en la pantalla \cdtIdRef{IUIBB 2}{Registrar información base para indicadores de biodiversidad}. \label{cuibb2:IngresarDatos}
    \UCpaso[\UCactor] Oprime el botón \cdtButton{Aceptar} en la pantalla \cdtIdRef{IUIBB 2}{Registrar información base para indicadores de biodiversidad} para confirmar el registro de la información base para indicadores de biodiversidad. \refTray{C}. \refTray{E}.    
    \UCpaso[\UCsist] Verifica que la escuela se encuentre en estado ``Información base en edición''. \refTray{A}. \label{cuibb2:VerificarRegistro}
    \UCpaso[\UCsist] Verifica que la fecha actual se encuentre dentro del periodo definido por la SMAGEM para realizar la operación. \refTray{B}.
    \UCpaso[\UCsist] Verifica que los datos ingresados proporcionen información correcta con base en la regla de negocio \cdtIdRef{RN-S1}{Información correcta}. \refTray{F}. \refTray{G}. \refTray{H}.
    \UCpaso[\UCsist] Registra la información base para indicadores de biodiversidad.
    \UCpaso[\UCsist] Muestra el mensaje \cdtIdRef{MSG1}{Operación realizada exitosamente} en la pantalla \cdtIdRef{IUIBB 1}{Administrar información base para indicadores de biodiversidad} para indicar al actor que el registro de la información se ha realizado exitosamente.    
 \end{UCtrayectoria}
 
    \begin{UCtrayectoriaA}[Fin del caso de uso]{A}{La escuela no se encuentra en un estado que permita realizar la operación.}
    \UCpaso[\UCsist] Muestra el mensaje \cdtIdRef{MSG28}{Operación no permitida por estado de la entidad} en la pantalla \cdtIdRef{IUIBB 1}{Administrar información base para indicadores de biodiversidad} indicando al actor que no puede realizar la operación debido a que la escuela no se encuentra en estado ``Información base en edición''. 
 \end{UCtrayectoriaA}

   \begin{UCtrayectoriaA}[Fin del caso de uso]{B}{La fecha actual se encuentra fuera del periodo definido por la SMAGEM para realizar la operación.}
    \UCpaso[\UCsist] Muestra el mensaje \cdtIdRef{MSG41}{Acción fuera del periodo} en la pantalla \cdtIdRef{IUIBB 1}{Administrar información base para indicadores de biodiversidad} indicando al actor que no puede realizar la operación debido a que la fecha actual se encuentra fuera del periodo definido por la SMAGEM para realizarla. 
 \end{UCtrayectoriaA}
 
  \begin{UCtrayectoriaA}{C}{El actor desea modificar la información base para indicadores de biodiversidad.}
    \UCpaso[\UCsist] Muestra el mensaje \cdtIdRef{MSG30}{Confirmar la modificación de un registro} en la pantalla emergente \cdtIdRef{IUIBB 2.1}{Registrar información base para indicadores de biodiversidad: Mensaje de confirmación} para que el actor confirme la modificación de la información base para indicadores de biodiversidad.
    \UCpaso[\UCactor] Oprime el botón \cdtButton{Aceptar} de la pantalla emergente \cdtIdRef{IUIBB 2.1}{Registrar información base para indicadores de biodiversidad: Mensaje de confirmación} confirmando la modificación de la información. \refTray{D}.
    \UCpaso[] Continúa con el paso \ref{cuibb2:VerificarRegistro} de la trayectoria principal.    
 \end{UCtrayectoriaA}
 
   \begin{UCtrayectoriaA}{D}{El actor desea cancelar la modificación de la información base para indicadores de biodiversidad.}
    \UCpaso[\UCactor] Solicita cancelar la modificación de la información oprimiendo el botón \cdtButton{Cancelar} de la pantalla emergente \cdtIdRef{IUIBB 2.1}{Administrar información base para indicadores de biodiversidad: Mensaje de confirmación}.
    \UCpaso[] Continúa con el paso \ref{cuibb2:IngresarDatos} de la trayectoria principal.    
 \end{UCtrayectoriaA}
 
    \begin{UCtrayectoriaA}{E}{El actor desea cancelar la operación.}
    \UCpaso[\UCactor] Solicita cancelar la operación oprimiendo el botón \cdtButton{Cancelar} en la pantalla \cdtIdRef{IUIBB 2}{Registrar información base para indicadores de biodiversidad}.
    \UCpaso[] Regresa a la pantalla \cdtIdRef{IUIBB 1}{Administrar información base para indicadores de biodiversidad}. 
    \end{UCtrayectoriaA}
  
    \begin{UCtrayectoriaA}{F}{El actor no ingresó un dato marcado como requerido.}    
    \UCpaso[\UCsist] Muestra el mensaje \cdtIdRef{MSG5}{Falta un dato requerido para efectuar la operación solicitada} en la pantalla \cdtIdRef{IUIBB 2}{Registrar información base para indicadores de biodiversidad} indicando que el registro de información base para indicadores de biodiversidad no puede realizarse debido a la falta de información requerida.
    \UCpaso[] Continúa con el paso \ref{cuibb2:IngresarDatos} de la trayectoria principal.     
    \end{UCtrayectoriaA}
 
        \begin{UCtrayectoriaA}{G}{El actor ingresó un tipo de dato incorrecto.}    
    \UCpaso[\UCsist] Muestra el mensaje \cdtIdRef{MSG6}{Formato incorrecto} en la pantalla \cdtIdRef{IUIBB 2}{Registrar información base para indicadores de biodiversidad} indicando que el registro de información base para indicadores de biodiversidad no puede realizarse debido a que la información ingresada no es correcta.
    \UCpaso[] Continúa con el paso \ref{cuibb2:IngresarDatos} de la trayectoria principal.     
    \end{UCtrayectoriaA}
    
            \begin{UCtrayectoriaA}{H}{El actor ingresó un dato que excede la longitud máxima.}    
    \UCpaso[\UCsist] Muestra el mensaje \cdtIdRef{MSG7}{Se ha excedido la longitud máxima del campo} en la pantalla \cdtIdRef{IUIBB 2}{Registrar información base para indicadores de biodiversidad} indicando que el registro de información base para indicadores de biodiversidad no puede realizarse debido a que la longitud del campo excede la longitud máxima definida.
    \UCpaso[] Continúa con el paso \ref{cuibb2:IngresarDatos} de la trayectoria principal.     
    \end{UCtrayectoriaA}