%!TEX encoding = UTF-8 Unicode

\begin{UseCase}{CUIBR 1}{Administrar información base para indicadores de residuos sólidos}
    {
	Los residuos sólidos son aquellos materiales desechados una vez que ha finalizado su vida útil, son desechos procedentes de materiales utilizados en la fabricación, transformación o utilización de bienes de consumo. Este caso de uso permite al actor administrar los residuos sólidos que se generan en la escuela y además sirve como punto de acceso para registrar, modificar o eliminar registros de residuos sólidos.
    }
    
    \UCitem{Versión}{1.0}
    \UCccsection{Administración de Requerimientos}
    \UCitem{Autor}{Jessica Stephany Becerril Delgado}
    \UCccitem{Evaluador}{}
    \UCitem{Operación}{Administración}
    \UCccitem{Prioridad}{Media}
    \UCccitem{Complejidad}{Media}
    \UCccitem{Volatilidad}{Alta}
    \UCccitem{Madurez}{Media}
    \UCitem{Estatus}{Terminado}
    \UCitem{Fecha del último estatus}{25 de Noviembre del 2014}
    
%% Copie y pegue este bloque tantas veces como revisiones tenga el caso de uso.
%% Esta sección la debe llenar solo el Revisor
% %--------------------------------------------------------
 	\UCccsection{Revisión Versión 0.1} % Anote la versión que se revisó.
% 	% FECHA: Anote la fecha en que se terminó la revisión.
 	\UCccitem{Fecha}{Fecha en que se termino la revisión} 
% 	% EVALUADOR: Coloque el nombre completo de quien realizó la revisión.
 	\UCccitem{Evaluador}{David Ortega Pacheco}
% 	% RESULTADO: Coloque la palabra que mas se apegue al tipo de acción que el analista debe realizar.
 	\UCccitem{Resultado}{Corregir}
% 	% OBSERVACIONES: Liste los cambios que debe realizar el Analista.
 	\UCccitem{Observaciones}{
 		\begin{UClist}
 			% PC: Petición de Cambio, describa el trabajo a realizar, si es posible indique la causa de la PC. Opcionalmente especifique la fecha en que considera razonable que se deba terminar la PC. No olvide que la numeración no se debe reiniciar en una segunda o tercera revisión.
 			\RCitem{PC1}{\TOCHK{Resumen, 3a línea, quitar coma y cambiar a:  ... en la escuela y además ... }}{Fecha de entrega}
 			\RCitem{PC2}{\TOCHK{Precondiciones, tomar en cuenta el estado de Edición}}{Fecha de entrega}
 			\RCitem{PC3}{\TOCHK{Salidas, la liga de Tipo, Residuo y Total semanal no lleva donde corresponde}}{Fecha de entrega}
 			\RCitem{PC4}{\TOCHK{Trayectoria Principal, Paso 1, la opción debe ser: Información base para indicadores}}{Fecha de entrega}
 			\RCitem{PC5}{\TOCHK{Salidas, Residuo debería ser Origen}}{Fecha de entrega} 			
 		\end{UClist}		
 	}
% %--------------------------------------------------------

    \UCsection{Atributos}
    \UCitem{Actor}{\cdtRef{actor:usuarioEscuela}{Coordinador del programa}}
    \UCitem{Propósito}{Administrar el registro, modificación y eliminación de los residuos sólidos que genera la escuela.}
    \UCitem{Entradas}{
	\begin{UClist}
	    \UCli Ninguna.
	\end{UClist}
    }
    \UCitem{Salidas}{
	\begin{UClist} 
	    \UCli \cdtRef{residuoSolido}{Residuo sólido}: \ioTabla{\cdtRef{residuoSolido:tipoDeResiduo}{Tipo}, \cdtRef{residuoSolido:residuo}{Residuo} y  \cdtRef{residuoSolido:cantidadSemanal}{Total semanal (kg/semanas)}}{que estén en el sistema}.
	\end{UClist}
    }

    \UCitem{Precondiciones}{
	\begin{UClist}
	    \UCli {\bf Interna:} Que la escuela se encuentre en estado \cdtRef{estado:infoBaseEdicion}{Información base en edición}.
	    \UCli {\bf Interna:} Que el periodo de registro de información base se encuentre vigente. 
	\end{UClist}
    }
    
    \UCitem{Postcondiciones}{
	\begin{UClist}
	    \UCli {\bf Interna:} Se podrá registrar un residuo sólido a través del caso de uso \cdtIdRef{CUIBR 2}{Registrar residuo sólido}.
	    \UCli {\bf Interna:} Se podrá modificar un residuo sólido a través del caso de uso \cdtIdRef{CUIBR 3}{Modificar residuo sólido}.
	    \UCli {\bf Interna:} Se podrá eliminar un residuo sólido a través del caso de uso \cdtIdRef{CUIBR 4}{Eliminar residuo sólido}.
	\end{UClist}
    }
    
    \UCitem{Reglas de negocio}{
    	\begin{UClist}
	    \UCli Ninguna.
	\end{UClist}
    }
    
    \UCitem{Errores}{
	\begin{UClist}
	    \UCli \cdtIdRef{MSG2}{No existe información registrada por el momento}: Se muestra en la pantalla \cdtIdRef{IUIBR 1}{Administrar información base para indicadores de residuos sólidos} indicando al actor que no existen registros de residuos sólidos en el sistema por el momento.
	    
	    \UCli \cdtIdRef{MSG28}{Operación no permitida por estado de la entidad}: Se muestra en la pantalla \cdtIdRef{IUIBR 1}{Administrar información base para indicadores de residuos sólidos} indicando al actor que no puede realizar la operación debido al estado en que se encuentra la escuela.
	    
	    \UCli \cdtIdRef{MSG41}{Acción fuera del periodo}: Se muestra en la pantalla \cdtIdRef{IUIBR 1}{Administrar información base para indicadores de residuos sólidos} para indicarle al actor que no puede realizar la operación debido a que la fecha actual se encuentra fuera del periodo definido por la SMAGEM para realizarla.
	\end{UClist}
    }

    \UCitem{Tipo}{Primario}

%    \UCitem{Fuente}{
%	\begin{UClist}
%	    \UCli Minuta de la reunión \cdtIdRef{M-17RT}{Reunión de trabajo}.
%	\end{UClist}
 %   }
\end{UseCase}

 \begin{UCtrayectoria}
    \UCpaso[\UCactor] Solicita administrar los residuos sólidos, seleccionando en el menú \cdtIdRef{MN2}{Menú del Coordinador del programa} la opción ``Información base para indicadores'' y posteriormente la opción ``Residuos sólidos''. 
    \UCpaso[\UCsist] Verifica que la escuela se encuentre en estado ``Información base en edición''. \refTray{A}.
    \UCpaso[\UCsist] Verifica que la fecha actual se encuentre dentro del periodo definido por la SMAGEM para realizar la operación. \refTray{B}.
    \UCpaso[\UCsist] Busca la información de los residuos sólidos registrados en el sistema. \refTray{C}.
    \UCpaso[\UCsist] Muestra la pantalla \cdtIdRef{IUIBR 1}{Administrar información base para indicadores de residuos sólidos}.
    \UCpaso[\UCactor] Administra los residuos sólidos a través de los botones \cdtButton{Registrar}, \botEdit y \botKo  . \label{cuibr1:Registrar}
 \end{UCtrayectoria}
 
 \begin{UCtrayectoriaA}[Fin del caso de uso]{A}{La escuela no se encuentra en un estado que permita realizar la operación.}
    \UCpaso[\UCsist] Muestra el mensaje \cdtIdRef{MSG28}{Operación no permitida por estado de la entidad} en la pantalla \cdtIdRef{IUIBR 1}{Administrar información base para indicadores de residuos sólidos} indicando al actor que no puede realizar la operación debido a que la escuela no se encuentra en estado ``Información base en edición''. 
 \end{UCtrayectoriaA}
 
 \begin{UCtrayectoriaA}[Fin del caso de uso]{B}{La fecha actual se encuentra fuera del periodo definido por la SMAGEM para realizar la operación.}
    \UCpaso[\UCsist] Muestra el mensaje \cdtIdRef{MSG41}{Acción fuera del periodo} en la pantalla \cdtIdRef{IUIBR 1}{Administrar información base para indicadores de residuos sólidos} indicando al actor que no puede realizar la operación debido a que la fecha actual se encuentra fuera del periodo definido por la SMAGEM para realizarla. 
 \end{UCtrayectoriaA}
 
  \begin{UCtrayectoriaA}[Fin del caso de uso]{C}{No hay registros de residuos sólidos para mostrar.}
    \UCpaso[\UCsist] Muestra el mensaje \cdtIdRef{MSG2}{No existe información registrada por el momento} en la pantalla \cdtIdRef{IUIBB 1}{Administrar información base para indicadores de residuos sólidos} indicando al actor que aún no hay residuos sólidos registrados. 
 \end{UCtrayectoriaA}


\subsection{Puntos de extensión}

\UCExtensionPoint
{El actor desea registrar un residuo sólido}
{ Paso \ref{cuibr1:Registrar} de la trayectoria principal}
{\cdtIdRef{CUIBR 2}{Registrar residuo sólido}}

\UCExtensionPoint
{El actor desea modificar un residuo sólido}
{ Paso \ref{cuibr1:Registrar} de la trayectoria principal}
{\cdtIdRef{CUIBR 3}{Modificar residuo sólido}}

\UCExtensionPoint
{El actor desea eliminar un residuo sólido}
{ Paso \ref{cuibr1:Registrar} de la trayectoria principal}
{\cdtIdRef{CUIBR 4}{Eliminar residuo sólido}}
