%!TEX encoding = UTF-8 Unicode

\begin{UseCase}{CUIBA 2}{Registrar información base para indicadores de agua}
    {
	  Este caso de uso permite al actor el registro o modificación de la información base que contribuya como referencia para los indicadores de agua, proporcionando una visión general del estado en que se encuentra una escuela con respecto al consumo y ahorro de agua en un periodo de tiempo definido. Una vez que los datos solicitados para el registro de la información base para indicadores de agua han sido ingresados, el sistema valida la información y esta queda registrada.
    }
    
    \UCitem{Versión}{1.0}
    \UCccsection{Administración de Requerimientos}
    \UCitem{Autor}{Jessica Stephany Becerril Delgado}
    \UCccitem{Evaluador}{}
    \UCitem{Operación}{Registro}
    \UCccitem{Prioridad}{Media}
    \UCccitem{Complejidad}{Media}
    \UCccitem{Volatilidad}{Media}
    \UCccitem{Madurez}{Media}
    \UCitem{Estatus}{Terminado}
    \UCitem{Fecha del último estatus}{19 de Noviembre del 2014}

%% Copie y pegue este bloque tantas veces como revisiones tenga el caso de uso.
%% Esta sección la debe llenar solo el Revisor
% %--------------------------------------------------------
 	\UCccsection{Revisión Versión 0.1} % Anote la versión que se revisó.
% 	% FECHA: Anote la fecha en que se terminó la revisión.
 	\UCccitem{Fecha}{24-nov-2014} 
% 	% EVALUADOR: Coloque el nombre completo de quien realizó la revisión.
 	\UCccitem{Evaluador}{David Ortega Pacheco}
% 	% RESULTADO: Coloque la palabra que mas se apegue al tipo de acción que el analista debe realizar.
 	\UCccitem{Resultado}{Corregir.}
% 	% OBSERVACIONES: Liste los cambios que debe realizar el Analista.
 	\UCccitem{Observaciones}{
 		\begin{UClist}
% 			% PC: Petición de Cambio, describa el trabajo a realizar, si es posible indique la causa de la PC. Opcionalmente especifique la fecha en que considera razonable que se deba terminar la PC. No olvide que la numeración no se debe reiniciar en una segunda o tercera revisión.
 			\RCitem{PC1}{\DONE{Resumen, no esta bien escrita la palabra: información, al inicio del párrafo}}{24-Nov-2014}
 			\RCitem{PC2}{\DONE{Resumen, al inicio del párrafo cambiar a: de la información ...}}{28-Nov-2014}
 			\RCitem{PC3}{\DONE{Propósito, otro propósito es el modificar}}{28-Nov-2014}
 			\RCitem{PC4}{\DONE{Entrada, Falta: Tipo de periodo}}{24-Nov-2014}
 			\RCitem{PC5}{\DONE{Entrada, consumo e importe anual y consumo e importe anual promedio son lo mismo, es decir, llevan a las mismas ligas y de acuerdo a la pantalla es consumo e importe anual promedio}}{24-Nov-2014}
 			\RCitem{PC6}{\DONE{Salida, liga rota en: Frecuencia}}{24-Nov-2014}
  			\RCitem{PC7}{\DONE{Salida, Falta: Tipo de periodo}}{24-Nov-2014}
 			\RCitem{PC8}{\DONE{Pantalla, comandos, cambiar a: registrar o modificar la ... }}{24-Nov-2014} 			
 			\RCitem{PC9}{\DONE{Liga rota en IUIBA 2.1 ...}}{24-Nov-2014}
			\RCitem{PC10}{\DONE{Trayectoria principal, en el paso 4 falta especificar sobre qué línea de acción es la información base para indicadores que se busca}}{24-Nov-2014}
 			\RCitem{PC5}{\TOCHK{Entrada, en Año no lleva a la información correcta}}{24-Nov-2014}
 		\end{UClist}		
 	}
% %--------------------------------------------------------

    \UCsection{Atributos}
    \UCitem{Actor}{\cdtRef{actor:usuarioEscuela}{Coordinador del programa}}
    \UCitem{Propósito}{Registrar o modificar la información base que permita conocer el estado en que se encuentra una escuela en cuestión de consumo y ahorro de agua.}
    \UCitem{Entradas}{
	\begin{UClist}
	    \UCli Información del consumo de agua:
	    \begin{itemize}
	    \item \cdtRef{consumo-agua:tipoAbastecimiento}{Tipos de abastecimiento de agua}: \ioCheckBox.
	    \item \cdtRef{consumo-agua:recibos}{Recibos de agua}: \ioRadioBoton.
	    \item \cdtRef{consumo-agua:tipoDePeriodo}{Tipo de periodo}: \ioSeleccionar.	    
	    \item \cdtRef{mensual:mes}{Mes}: \ioSeleccionar.
	    \item \cdtRef{bimestral:bimestre}{Bimestre}: \ioSeleccionar.
	    \item \cdtRef{semestral:semestre}{Semestre}: \ioSeleccionar.
	    \item \cdtRef{periodo-consumo:Anio}{Año}: \ioSeleccionar.	    
	    \item \cdtRef{periodo-consumo:consumo}{Consumo por tipo de periodo}: \ioEscribir.
	    \item \cdtRef{periodo-consumo:importe}{Importe por tipo de periodo}: \ioEscribir.
	    \end{itemize}
	\end{UClist}
    }

    \UCitem{Salidas}{
	\begin{UClist}
	    \UCli Información del consumo de agua:
	    \begin{itemize}
	    \item \cdtRef{consumo-agua:tipoAbastecimiento}{Tipos de abastecimiento de agua}: \ioObtener.
	    \item \cdtRef{consumo-agua:recibos}{Recibos de agua}: \ioObtener.
	    \item \cdtRef{consumo-agua:tipoDePeriodo}{Tipo de periodo}: \ioObtener.	    
	    \item \cdtRef{mensual:mes}{Mes}: \ioObtener.
	    \item \cdtRef{bimestral:bimestre}{Bimestre}: \ioObtener.
	    \item \cdtRef{semestral:semestre}{Semestre}: \ioObtener.
	    \item \cdtRef{periodo-consumo:Anio}{Año}: \ioObtener.	    
	    \item \cdtRef{periodo-consumo:consumo}{Consumo por tipo de periodo}: \ioObtener.
	    \item \cdtRef{periodo-consumo:importe}{Importe por tipo de periodo}: \ioObtener.
	    
	    \item Consumo total: \ioCalcular \cdtIdRef{RN-N12}{Calcular consumo total}.
	    \item Importe total: \ioCalcular \cdtIdRef{RN-N13}{Calcular importe total}.
	    \end{itemize}

	    \UCli \cdtIdRef{MSG1}{Operación realizada exitosamente:} Se muestra en la pantalla \cdtIdRef{IUIBA 1}{Administrar información base para indicadores de agua} cuando el registro de la información base para indicadores de agua se ha realizado correctamente.
	    
	    \UCli \cdtIdRef{MSG30}{Confirmar la modificación de un registro}: Se muestra en la pantalla emergente \cdtIdRef{IUIBA 2.1}{Registrar información base para indicadores de agua: Mensaje de confirmación} para indicar al actor que al guardar los cambios realizados la información previa se perderá.
	\end{UClist}
    }

    \UCitem{Precondiciones}{
	\begin{UClist}
	    \UCli {\bf Interna:} Que la escuela se encuentre en estado \cdtRef{estado:infoBaseEdicion}{Información base en edición}.
	    \UCli {\bf Interna:} Que el periodo de registro de información base se encuentre vigente. 
	\end{UClist}
    }
    
    \UCitem{Postcondiciones}{
	\begin{UClist}
	    \UCli {\bf Interna:} Se podrá modificar la información base definida para los indicadores de agua a través del caso de uso \cdtIdRef{CUIBA 2}{Registrar información base para indicadores de agua}.
	\end{UClist}
    }
    
    \UCitem{Reglas de negocio}{
    	\begin{UClist}
	    \UCli \cdtIdRef{RN-S1}{Información correcta}: Verifica que la información introducida sea correcta.
	    \UCli \cdtIdRef{RN-N12}{Calcular consumo total}: Calcula el consumo total de agua con base en la información ingresada de los recibos.
	    \UCli \cdtIdRef{RN-N13}{Calcular importe total}: Calcula el importe total de agua con base en la información ingresada de los recibos.
	\end{UClist}
    }
    
    \UCitem{Errores}{
	\begin{UClist}
	    \UCli \cdtIdRef{MSG4}{No se encontró información sustantiva}: Se muestra en la pantalla \cdtIdRef{IUIBA 1}{Administrar información base para indicadores de agua} cuando el sistema no cuenta con información en los catálogos de bimestre o año.
	    
	    \UCli \cdtIdRef{MSG5}{Falta un dato requerido para efectuar la operación solicitada}: Se muestra en la pantalla \cdtIdRef{IUIBA 2}{Registrar información base para indicadores de agua} cuando no se ha ingresado un dato marcado como requerido.
	    
	    \UCli \cdtIdRef{MSG6}{Formato incorrecto}: Se muestra en la pantalla \cdtIdRef{IUIBA 2}{Registrar información base para indicadores de agua} cuando el tipo de dato ingresado no cumple con el tipo de dato solicitado en el campo.
	    
	    \UCli \cdtIdRef{MSG7}{Se ha excedido la longitud máxima del campo}: Se muestra en la pantalla \cdtIdRef{IUIBA 2}{Registrar información base para indicadores de agua} cuando se ha excedido la longitud de alguno de los campos.	    
	    
	    \UCli \cdtIdRef{MSG28}{Operación no permitida por estado de la entidad}: Se muestra en la pantalla \cdtIdRef{IUIBA 1}{Administrar información base para indicadores de agua} indicando al actor que no se puede realizar la operación debido al estado en que se encuentra la escuela.
	    
	    \UCli \cdtIdRef{MSG41}{Acción fuera del periodo}: Se muestra sobre la pantalla en la pantalla \cdtIdRef{IUIBA 1}{Administrar información base para indicadores de agua} para indicarle al actor que no puede realizar la operación debido a que la fecha actual se encuentra fuera del periodo definido por la SMAGEM para realizarla.
	\end{UClist}
    }

    \UCitem{Tipo}{Secundario, extiende del caso de uso \cdtIdRef{CUIBA 1}{Administrar información base para indicadores de agua}.}

%    \UCitem{Fuente}{
%	\begin{UClist}
%	    \UCli Minuta de la reunión \cdtIdRef{M-17RT}{Reunión de trabajo}.
%	\end{UClist}
%    }
\end{UseCase}

\begin{UCtrayectoria}
    \UCpaso[\UCactor] Solicita registrar la información base para indicadores de agua oprimiendo el botón \botEdit de la pantalla \cdtIdRef{IUIBA 1}{Administrar información base para indicadores de agua}.
    \UCpaso[\UCsist] Verifica que la escuela se encuentre en estado ``Información base en edición''. \refTray{A}.
    \UCpaso[\UCsist] Verifica que la fecha actual se encuentre dentro del periodo definido por la SMAGEM para realizar la operación. \refTray{B}.
    \UCpaso[\UCsist] Busca la información referente a los catálogos tipo de periodo, mes, bimestre, semestre y año. \refTray{C}.
    \UCpaso[\UCsist] Busca la información previamente registrada referente a la información base para indicadores de agua.
    \UCpaso[\UCsist] Muestra la pantalla \cdtIdRef{IUIBA 2}{Registrar información base para indicadores de agua} por medio de la cual se realizará el registro de información base para indicadores de agua.
    \UCpaso[\UCactor] Ingresa la información referente a la información base para indicadores de agua en la pantalla \cdtIdRef{IUIBA 2}{Registrar información base para indicadores de agua}. \label{cuiba2:IngresarDatos}
    \UCpaso[\UCactor] Oprime el botón \cdtButton{Aceptar} en la pantalla \cdtIdRef{IUIBA 2}{Registrar información base para indicadores de agua} para confirmar el registro de la información base para indicadores de agua. \refTray{D}. \refTray{F}.    
     \UCpaso[\UCsist] Verifica que la escuela se encuentre en estado ``Información base en edición''. \refTray{A}. \label{cuiba2:VerificarRegistro}
    \UCpaso[\UCsist] Verifica que la fecha actual se encuentre dentro del periodo definido por la SMAGEM para realizar la operación. \refTray{B}.
    \UCpaso[\UCsist] Verifica que los datos ingresados proporcionen información correcta con base en la regla de negocio \cdtIdRef{RN-S1}{Información correcta}. \refTray{G}. \refTray{H}. \refTray{I}. 
    \UCpaso[\UCsist] Registra la información base para indicadores de agua.
    \UCpaso[\UCsist] Muestra el mensaje \cdtIdRef{MSG1}{Operación realizada exitosamente} en la pantalla \cdtIdRef{IUIBA 1}{Administrar información base para indicadores de agua} para indicar al actor que el registro de la información se ha realizado exitosamente.    
 \end{UCtrayectoria}
 
   \begin{UCtrayectoriaA}[Fin del caso de uso]{A}{La escuela no se encuentra en un estado que permita realizar la operación.}
    \UCpaso[\UCsist] Muestra el mensaje \cdtIdRef{MSG28}{Operación no permitida por estado de la entidad} en la pantalla \cdtIdRef{IUIBA 1}{Administrar información base para indicadores de agua} indicando al actor que no puede realizar la operación debido a que la escuela no se encuentra en estado ``Información base en edición''. 
 \end{UCtrayectoriaA}

   \begin{UCtrayectoriaA}[Fin del caso de uso]{B}{La fecha actual se encuentra fuera del periodo definido por la SMAGEM para realizar la operación.}
    \UCpaso[\UCsist] Muestra el mensaje \cdtIdRef{MSG41}{Acción fuera del periodo} en la pantalla \cdtIdRef{IUIBA 1}{Administrar información base para indicadores de agua} indicando al actor que no puede realizar la operación debido a que la fecha actual se encuentra fuera del periodo definido por la SMAGEM para realizarla. 
 \end{UCtrayectoriaA}
 
  \begin{UCtrayectoriaA}[Fin del caso de uso]{C}{No existe información base en los catálogos de tipo de periodo, mes, bimestre, semestre o año.}
    \UCpaso[\UCsist] Muestra el mensaje \cdtIdRef{MSG4}{No se encontró información sustantiva} en la pantalla \cdtIdRef{IUIBA 1}{Administrar información base para indicadores de agua} indicando al actor que no puede registrar la información base para indicadores de agua debido a que no se cuenta con información sustantiva para los catálogos de tipo de periodo, mes, bimestre, semestre y año.
 \end{UCtrayectoriaA}
 
  \begin{UCtrayectoriaA}{D}{El actor desea modificar la información base para indicadores de agua.}
    \UCpaso[\UCsist] Muestra el mensaje \cdtIdRef{MSG30}{Confirmar la modificación de un registro} en la pantalla emergente \cdtIdRef{IUIBA 2.1}{Registrar información base para indicadores de agua: Mensaje de confirmación} para que el actor confirme la modificación de la información base para indicadores de agua.
    \UCpaso[\UCactor] Oprime el botón \cdtButton{Aceptar} de la pantalla emergente \cdtIdRef{IUIBA 2.1}{Registrar información base para indicadores de agua: Mensaje de confirmación} confirmando la modificación de la información. \refTray{D}.
    \UCpaso[] Continúa con el paso \ref{cuiba2:VerificarRegistro} de la trayectoria principal.    
 \end{UCtrayectoriaA}
 
   \begin{UCtrayectoriaA}{E}{El actor desea cancelar la modificación de la información base para indicadores de agua.}
    \UCpaso[\UCactor] Solicita cancelar la modificación de la información oprimiendo el botón \cdtButton{Cancelar} de la pantalla emergente \cdtIdRef{IUIBA 2.1}{Administrar información base para indicadores de agua: Mensaje de confirmación}.
    \UCpaso[] Continúa con el paso \ref{cuiba2:IngresarDatos} de la trayectoria principal.    
 \end{UCtrayectoriaA}
 
    \begin{UCtrayectoriaA}[Fin del caso de uso]{F}{El actor desea cancelar la operación.}
    \UCpaso[\UCactor] Solicita cancelar la operación oprimiendo el botón \cdtButton{Cancelar} en la pantalla \cdtIdRef{IUIBA 2}{Registrar información base para indicadores de agua}.
    \UCpaso[] Regresa a la pantalla \cdtIdRef{IUIBA 1}{Administrar información base para indicadores de agua}. 
    \end{UCtrayectoriaA}
  
    \begin{UCtrayectoriaA}{G}{El actor no ingresó un dato marcado como requerido.}    
    \UCpaso[\UCsist] Muestra el mensaje \cdtIdRef{MSG5}{Falta un dato requerido para efectuar la operación solicitada} en la pantalla \cdtIdRef{IUIBA 2}{Registrar información base para indicadores de agua} indicando que el registro de información base para indicadores de agua no puede realizarse debido a la falta de información requerida.
    \UCpaso[] Continúa con el paso \ref{cuiba2:IngresarDatos} de la trayectoria principal.     
    \end{UCtrayectoriaA}
 
        \begin{UCtrayectoriaA}{H}{El actor ingresó un tipo de dato incorrecto.}    
    \UCpaso[\UCsist] Muestra el mensaje \cdtIdRef{MSG6}{Formato incorrecto} en la pantalla \cdtIdRef{IUIBA 2}{Registrar información base para indicadores de agua} indicando que el registro de información base para indicadores de agua no puede realizarse debido a que la información ingresada no es correcta.
    \UCpaso[] Continúa con el paso \ref{cuiba2:IngresarDatos} de la trayectoria principal.     
    \end{UCtrayectoriaA}
    
            \begin{UCtrayectoriaA}{I}{El actor ingresó un dato que excede la longitud máxima.}    
    \UCpaso[\UCsist] Muestra el mensaje \cdtIdRef{MSG7}{Se ha excedido la longitud máxima del campo} en la pantalla \cdtIdRef{IUIBA 2}{Registrar información base para indicadores de agua} indicando que el registro de información base para indicadores de agua no puede realizarse debido a que la longitud del campo excede la longitud máxima definida.
    \UCpaso[] Continúa con el paso \ref{cuiba2:IngresarDatos} de la trayectoria principal.     
    \end{UCtrayectoriaA}